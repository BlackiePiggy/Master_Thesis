% !TEX root = ../main.tex

\chapter{GNSS事件对定位性能影响评估}

GNSS事件带来的无论是信号层面的质量波动,还是导航电文参数的异常,其影响评估的最终落脚点都在于用户终端的定位性能。对于高精度定位用户而言,理解并量化这些事件对最终坐标解算精度的影响至关重要。GNSS事件种类繁多,其中星载产品异常通常会导致卫星信号中断或被完好性监测系统剔除,其影响具有显性和不可逆的特征。相比之下,弹性功率事件具有更强的隐蔽性。正如第二章所述,弹性功率调整期间,卫星的伪距硬件延迟发生显著变化,进而引起DCB参数的漂移。在 PPP 中,DCB是解决观测方程秩亏、维持参数估计稳定性的关键修正量。如果在使用精密钟差产品时未正确处理弹性功率引发的DCB变化,将直接导致定位模型的系统性偏差,影响收敛时间和定位精度。本章主要聚焦于弹性功率事件对精密定位性能的影响评估。首先,本章将从PPP的函数模型入手,详细推导双频IF模型与非差非组合模型,从理论层面探明DCB在观测方程中的具体作用机制及其对定位解算的影响路径。随后,设计针对弹性功率期间的DCB估计策略,并选取典型的弹性功率事件时段进行实验。最后,通过对比分析不同处理策略下的定位结果以评估弹性功率事件对用户端精密定位性能的具体影响。

\section{PPP原始观测方程及函数模型}
 PPP 是指利用全球分布的参考站网解算的精密卫星轨道和钟差产品,对单台GNSS接收机的非差观测值进行处理,以获得高精度位置坐标、接收机钟差及大气延迟参数的技术。构建严密的函数模型是实现PPP高精度解算的基础。

\subsection{原始观测方程}
GNSS的基本观测值包括伪距和载波相位。假设接收机为 $r$,卫星为 $s$,在频率 $j$ ($j=1, 2, ...$) 上的原始观测方程可表示为:
\begin{equation}
\left\{
\begin{aligned}
P_{r,j}^s &= \rho_r^s + c(dt_r - dt^s) + T_r^s + I_{r,j}^s + d_{r,j} - d_{j}^s + \epsilon_{P,j}\\
L_{r,j}^s &= \rho_r^s + c(dt_r - dt^s) + T_r^s - I_{r,j}^s + \lambda_j N_{r,j}^s + b_{r,j} - b_{j}^s + \epsilon_{L,j}
\end{aligned}
\right.
\label{eq:gnss_raw_obs}
\end{equation}
其中:
\begin{itemize}
    \item $P_{r,j}^s$ 和 $L_{r,j}^s$ 分别为伪距观测值和载波相位观测值(单位:米);
    \item $\rho_r^s$ 为卫星天线相位中心至接收机天线相位中心的几何距离;
    \item $c$ 为真空中的光速;
    \item $dt_r$ 和 $dt^s$ 分别为接收机钟差和卫星钟差;
    \item $T_r^s$ 为对流层延迟(包含干分量和湿分量);
    \item $I_{r,j}^s$ 为频率 $j$ 处的电离层延迟一阶项;
    \item $\lambda_j$ 为载波波长;
    \item $N_{r,j}^s$ 为整周模糊度;
    \item $d_{r,j}$ 和 $d_{j}^s$ 分别为接收机端和卫星端的伪距硬件延迟;
    \item $b_{r,j}$ 和 $b_{j}^s$ 分别为接收机端和卫星端的相位硬件延迟;
    \item $\epsilon_{P,j}$ 和 $\epsilon_{L,j}$ 分别包含了多路径效应及测量噪声。
\end{itemize}

上述方程中,几何距离 $\rho_r^s$ 实际上是关于接收机位置坐标 $(x_r, y_r, z_r)$ 的非线性函数。在实际平差计算中,通常利用泰勒级数展开将其线性化。此外,相对论效应、萨格纳克效应(Sagnac effect)、天线相位中心偏差(PCO)及变化(PCV)、潮汐负荷等误差项需在进入平差模型前通过模型改正予以消除。

\subsection{双频IF组合模型}
电离层延迟是GNSS信号穿过大气层时受自由电子含量影响产生的最大误差源之一,其量级可达数十米。由于电离层是一种色散介质,其延迟量与信号频率的平方成反比,这为利用双频观测值消除电离层影响提供了物理基础。双频无电离层组合(Ionosphere-Free, IF)模型,旨在通过对两个不同频率的观测值进行特定的线性组合,以数学方式消除一阶电离层延迟的影响。该模型无需引入外部电离层先验信息即可实现分米级甚至厘米级的定位精度,因此成为精密单点定位中最经典且应用最为广泛的函数模型。

基于电离层的色散特性,构建消电离层组合的关键在于确定组合系数。假设接收机使用频率为 $f_1$ 和 $f_2$ 的双频信号,为了消除一阶电离层项 $I_{r,j}^s$,同时保留几何距离 $\rho_r^s$ 的几何尺度不变,根据线性组合原理,可定义双频消电离层组合观测值 $P_{IF}$ 和 $L_{IF}$。设组合系数分别为 $\alpha_{12}$ 和 $\beta_{12}$,其数学表达式及约束条件如下:
\begin{equation}
\left\{
\begin{aligned}
&\alpha_{12} = \frac{f_1^2}{f_1^2 - f_2^2}, \quad \beta_{12} = - \frac{f_2^2}{f_1^2 - f_2^2} \\
&P_{IF} = \alpha_{12} P_{r,1}^s + \beta_{12} P_{r,2}^s \\
&L_{IF} = \alpha_{12} L_{r,1}^s + \beta_{12} L_{r,2}^s
\end{aligned}
\right.
\end{equation}
将原始观测方程代入上式,并忽略高阶电离层残差项,可得双频IF组合的线性化观测方程:
\begin{equation}
\left\{
\begin{aligned}
P_{IF} &= \rho_r^s + c \cdot \bar{dt}_r - c \cdot \bar{dt}^s + T_r^s + \varepsilon_{P,IF} \\
L_{IF} &= \rho_r^s + c \cdot \bar{dt}_r - c \cdot \bar{dt}^s + T_r^s + \lambda_{IF} \bar{N}_{IF} + \varepsilon_{L,IF}
\end{aligned}
\right.
\end{equation}
式中,$\lambda_{IF}$ 为组合波长。

双频IF模型的主要优势在于其能够在观测方程层面严格消除电离层一阶延迟,从而显著降低系统误差对定位结果的影响,尤其适用于全球尺度、长基线以及电离层活动较强区域的高精度PPP解算。此外,该模型避免了显式估计电离层参数,使函数模型更加简洁,有利于提高参数估计的稳定性。然而,双频IF组合也存在一定的不足。由于线性组合放大了观测噪声,其组合后的伪距和载波相位噪声水平均高于原始单频观测值;同时,IF组合导致模糊度失去整数特性,限制了传统整周模糊度固定技术的直接应用,从而在一定程度上影响收敛速度。

需要特别说明的是,线性组合过程不可避免地对模型中的硬件延迟与模糊度参数产生了重组效应。方程中的接收机钟差 $\bar{dt}_r$ 实际上吸收了接收机端的消电离层组合伪距硬件延迟;而卫星钟差 $\bar{dt}^s$ 则对应于IGS提供的精密钟差产品。若不加以处理,将对钟差参数及定位结果产生系统性偏移。因此,在基于双频IF伪距观测进行精密单点定位时,通常需要引入由分析中心提供的卫星端DCB产品,或通过模型重参数化将其吸收到接收机钟差中进行处理。相比之下,在仅使用载波相位IF组合或采用无电离层相位主导的PPP模型时,DCB对最终定位结果的影响相对较弱,可根据具体应用需求进行取舍。由于IGS精密钟差通常是基于双频消电离层组合生成的,其数值中已包含了卫星端的伪距硬件延迟,因此在采用与IGS产品一致的频率对(如在GPS系统中是L1/L2,也写作C1C/C2W,在BDS系统中则是C2I/C6I)进行解算时,无需额外引入DCB改正。然而,这一参数重组导致相位方程中的模糊度参数 $\bar{N}_{IF}$ 不再具有整数特性,而是退化为包含整周模糊度及收发两端未被钟差吸收的硬件延迟偏差的浮点数,这也是传统PPP通常获得浮点解的原因。


\subsection{非差非组合模型}
双频IF组合模型中,虽然通过线性组合有效地消除了电离层一阶项的影响,但该方法存在噪声放大和电离层信息丢失两个显著的理论缺陷。非差非组合(Uncombined and Undifferenced, UC/UD)模型直接利用原始的频率观测值构建方程,将电离层延迟作为待估参数进行估计,从而避免了观测值组合带来的噪声放大,并保留了所有原始观测信息。非差非组合模型在灵活性、多系统融合以及大气建模方面具有的显著优势使其成为近年来GNSS高精度定位领域的研究热点。

根据观测信号的物理传输机制,假设接收机为 $r$,观测到的卫星为 $s$,对于频率 $f_1$ 和 $f_2$,顾及电离层延迟、对流层延迟及相关硬件延迟偏差,其非差非组合原始观测方程组可表述为:
\begin{equation}
\left\{
\begin{aligned}
P^{s}_{r,1} &=
\rho^{s}_{r}
+ c\,dt_r
- c\,dt^{s}
+ m^{s}_{r,d} Z_{r,d}
+ m^{s}_{r,w} Z_{r,w}
+ I^{s}_{r,1}
+ b_{r,1}
- b^{s}_{1}
+ \varepsilon^{s}_{1,P}, \\
P^{s}_{r,2} &=
\rho^{s}_{r}
+ c\,dt_r
- c\,dt^{s}
+ m^{s}_{r,d} Z_{r,d}
+ m^{s}_{r,w} Z_{r,w}
+ \gamma_2 I^{s}_{r,1}
+ b_{r,2}
- b^{s}_{2}
+ \varepsilon^{s}_{2,P}, \\
L^{s}_{r,1} &=
\rho^{s}_{r}
+ c\,dt_r
- c\,dt^{s}
+ m^{s}_{r,d} Z_{r,d}
+ m^{s}_{r,w} Z_{r,w}
- I^{s}_{r,1}
+ \lambda^{s}_{1}
\left(
N^{s}_{r,1}
+ B_{r,1}
- B^{s}_{1}
\right)
+ \varepsilon^{s}_{1,\varphi}, \\
L^{s}_{r,2} &=
\rho^{s}_{r}
+ c\,dt_r
- c\,dt^{s}
+ m^{s}_{r,d} Z_{r,d}
+ m^{s}_{r,w} Z_{r,w}
- \gamma_2 I^{s}_{r,1}
+ \lambda^{s}_{2}
\left(
N^{s}_{r,2}
+ B_{r,2}
- B^{s}_{2}
\right)
+ \varepsilon^{s}_{2,\varphi}.
\end{aligned}
\right.
\label{eq:gnss_obs}
\end{equation}
其中:
\begin{itemize}
  \item $P^{s}_{r,j},\, L^{s}_{r,j}$:分别为频率 $j\,(j=1,2)$ 的伪距(单位:m)和载波相位观测值(单位:m);
  \item $\rho^{s}_{r}$:卫星至接收机的几何距离;
  \item $c$:真空中的光速;
  \item $dt_r,\, dt^{s}$:分别为接收机钟差和卫星钟差;
  \item $Z_{r,d},\, Z_{r,w}$:分别为接收机天顶对流层干延迟和湿延迟;
  \item $m^{s}_{r,d},\, m^{s}_{r,w}$:分别为对应的干分量和湿分量投影函数;
  \item $I^{s}_{r,1}$:$f_1$ 频率上的视线方向电离层延迟,
        $\gamma_2=(f_1/f_2)^2$ 为电离层频率转换因子;
  \item $b_{r,j},\, b^{s}_{j}$:分别为接收机端和卫星端在频率 $j$ 上的伪距硬件延迟(单位:m);
  \item $\lambda^{s}_{j}$:载波波长;
  \item $N^{s}_{r,j}$:频率 $j$ 上的整周模糊度(单位:周,cycle);
  \item $B_{r,j},\, B^{s}_{j}$:分别为接收机端和卫星端在频率 $j$ 上的载波相位硬件延迟(单位:周,cycle);
  \item $\varepsilon^{s}_{j,P},\, \varepsilon^{s}_{j,\varphi}$:包含多路径效应及测量噪声的残差项。
\end{itemize}


由于接收机钟差与接收机伪距硬件延迟、卫星钟差与卫星伪距硬件延迟、以及模糊度参数与相位硬件延迟之间存在线性相关性,无法直接分离,在参数估计过程中会面临严重的秩亏问题。因此,必须引入精密星历和钟差产品,并对参数进行重组。

解决模型秩亏的关键在于处理卫星钟差基准的不一致性以及接收机端参数的重新定义。由于IGS提供的精密卫星钟差产品 $dt^s_{IGS}$ 通常是基于双频消电离层组合(IF)生成的,该钟差内含了卫星端P1与P2码硬件延迟的线性组合(即 $c \cdot dt^s_{IGS} = c \cdot dt^s + \alpha_{12} b_1^s + \beta_{12} b_2^s$)。若直接将其应用于非差非组合模型,会导致模型物理意义不符,因此必须引入卫星端DCB进行修正。定义卫星端DCB为 $DCB^s_{P1P2} = b_1^s - b_2^s$,利用该参数可将IF基准的精密钟差还原至原始频率层面,从而消除卫星端硬件延迟的影响。同时,对于对流层延迟,通常采用Saastamoinen模型结合实测气象参数或GPT模型精确计算干延迟 $Z_{r,d}$ 并从观测值中扣除,仅保留湿延迟 $Z_{r,w}$ 作为待估参数。

在完成卫星端与大气模型的修正后,需进一步通过参数重组解决接收机端的秩亏问题。具体的重组策略是将接收机端 $f_1$ 频率的伪距硬件延迟 $b_{r,1}$ 吸收至接收机钟差参数中,形成广义接收机钟差 $\overline{dt}_r$;同时,电离层参数 $\overline{I}_{r,1}^s$ 也会相应吸收接收机及卫星端的DCB,变为“有偏”电离层参数;而模糊度参数 $\overline{N}_{r,j}^s$ 则吸收了相位与伪距硬件延迟的残余项,转变为浮点模糊度。基于此策略,最终推导出的满秩线性化观测方程组如下:
\begin{equation}
\left\{
\begin{aligned}
p^{s}_{r,1} &=
- \mathbf{u}^{s}_{r} \mathbf{x}
+ c \cdot \overline{dt}_r
+ m^{s}_{r,w} Z_{r,w}
+ \overline{I}^{s}_{r,1}
+ \varepsilon^{s}_{1,P}, \\
p^{s}_{r,2} &=
- \mathbf{u}^{s}_{r} \mathbf{x}
+ c \cdot \overline{dt}_r
+ m^{s}_{r,w} Z_{r,w}
+ \gamma_2 \overline{I}^{s}_{r,1}
- c \cdot \overline{DCB}^{s}_{P1P2}
+ \varepsilon^{s}_{2,P}, \\
l^{s}_{r,1} &=
- \mathbf{u}^{s}_{r} \mathbf{x}
+ c \cdot \overline{dt}_r
+ m^{s}_{r,w} Z_{r,w}
- \overline{I}^{s}_{r,1}
+ \lambda_1 \overline{N}^{s}_{r,1}
+ \varepsilon^{s}_{1,\varphi}, \\
l^{s}_{r,2} &=
- \mathbf{u}^{s}_{r} \mathbf{x}
+ c \cdot \overline{dt}_r
+ m^{s}_{r,w} Z_{r,w}
- \gamma_2 \overline{I}^{s}_{r,1}
+ \lambda_2 \overline{N}^{s}_{r,2}
+ \varepsilon^{s}_{2,\varphi}.
\end{aligned}
\right.
\label{eq:undiff_uncomb}
\end{equation}

在该模型中,待估参数向量定义为 $\mathbf{X} = \left[ \mathbf{x}, c \cdot \overline{dt}_r, Z_{r,w}, \overline{I}_{r,1}^s, \overline{N}_{r,1}^s, \overline{N}_{r,2}^s \right]^T$。通过上述线性化方程可以看出,非差非组合模型成功实现了在利用外部DCB产品修正卫星端偏差的前提下,利用卡尔曼滤波算法同时解算出接收机位置坐标、接收机钟差、对流层湿延迟、视线方向电离层延迟以及浮点模糊度。

\subsection{单频模型}

随着低成本GNSS芯片与物联网技术的发展,单频精密单点定位(Single-Frequency PPP, SF-PPP)在消费级市场中的应用日益广泛。与双频模型相比,单频模型仅使用单一频率(通常为$f_1$)的观测值,不仅面临严重的电离层延迟干扰,还存在观测值频率与精密钟差产品基准频率不一致的问题,这使得DCB的处理在SF-PPP中显得尤为关键。

对于单频接收机,其在 $f_1$ 频率上的原始观测方程为:
\begin{equation}
\left\{
\begin{aligned}
P_{r,1}^s &=
\rho_r^s
+ c \left( dt_r - dt^s \right)
+ T_r^s
+ I_{r,1}^s
+ d_{r,1}
- d_1^s
+ \varepsilon_{P,1}, \\
L_{r,1}^s &=
\rho_r^s
+ c \left( dt_r - dt^s \right)
+ T_r^s
- I_{r,1}^s
+ \lambda_1 N_{r,1}^s
+ b_{r,1}
- b_1^s
+ \varepsilon_{L,1}.
\end{aligned}
\right.
\end{equation}


在SF-PPP解算策略中,通常采用IGS分析中心提供的 GIM 来修正一阶电离层延迟 $I_{r,1}^s$,或采用半和组合(GRAPHIC)消除电离层影响。若采用GIM模型修正策略,模型中的核心难点在于卫星钟差的代入。IGS提供的精密卫星钟差产品 $dt^s_{IGS}$ 是基于双频消电离层组合(IF)定义的。若将该钟差直接代入单频 $f_1$ 的观测方程中,必须考虑 $f_1$ 频率的硬件延迟与双频组合硬件延迟之间的差异。将 $c \cdot dt^s = c \cdot dt^s_{IGS} - (\alpha_{12} b_1^s + \beta_{12} b_2^s)$ 代入伪距方程,并整理可得:
\begin{equation}
\begin{aligned}
P_{r,1}^s &=
\rho_r^s
+ c \cdot dt_r
- \left[
c \cdot dt^s_{\mathrm{IGS}}
-
\left(
\alpha_{12} b_1^s
+ \beta_{12} b_2^s
\right)
\right]
+ T_r^s
+ I_{r,1}^s(\mathrm{GIM})
+ d_{r,1}
- d_1^s
+ \varepsilon_{P,1} \\
&=
\rho_r^s
+ c \cdot \overline{dt}_r
- c \cdot dt^s_{\mathrm{IGS}}
+ T_r^s
+ I_{r,1}^s(\mathrm{GIM})
-
\left(
d_1^s
- \alpha_{12} b_1^s
- \beta_{12} b_2^s
\right)
+ \varepsilon_{P,1}.
\end{aligned}
\end{equation}



假设卫星端的伪距硬件延迟与相位硬件延迟在量级上存在差异但在此处仅考虑伪距偏差(即近似认为 $d^s \approx b^s$ 以简化推导,或严格按照DCB定义),利用系数关系 $\alpha_{12} + \beta_{12} = 1$,上式中卫星端的残留硬件延迟项可化简为与DCB相关的形式。最终,引入卫星端 DCB($DCB_{P1P2}^s$)修正后的SF-PPP线性化观测方程为:
\begin{equation}
\left\{
\begin{aligned}
p_{r,1}^s &=
- \mathbf{u}_r^s \, \mathbf{x}
+ c \cdot \overline{dt}_r
+ m_w^s Z_{r,w}
+ I_{r,1}^s(\mathrm{GIM})
+ \frac{f_2^2}{f_1^2 - f_2^2}
\, c \cdot DCB_{P1P2}^s
+ \varepsilon_{P,1}, \\
l_{r,1}^s &=
- \mathbf{u}_r^s \, \mathbf{x}
+ c \cdot \overline{dt}_r
+ m_w^s Z_{r,w}
- I_{r,1}^s(\mathrm{GIM})
+ \lambda_1 \overline{N}_1^s
+ \varepsilon_{L,1}.
\end{aligned}
\right.
\label{eq:sf_ppp}
\end{equation}


从式 (\ref{eq:sf_ppp}) 可以清晰地看到,SF-PPP模型在理论上对卫星端DCB产品具有刚性依赖。在常规情况下,DCB被认为是极其稳定的;然而,在发生弹性功率事件时,卫星播发的伪距信号特性发生改变,导致真实的硬件延迟发生漂移。如果此时用户端仍使用事件发生前(或全天解算)的静态DCB产品进行修正,上述方程中的DCB改正项将与当前历元的真实偏差不符,产生的残差将直接被接收机钟差吸收或被错误地分配到位置参数中,从而导致定位精度的恶化。因此,利用SF-PPP模型评估弹性功率期间DCB变化对定位的影响,能够直观地反映出该类事件对低成本用户端的潜在威胁。


\section{数据集与实验设计}
% ================== 思路 ===================
% 前面分析了不同PPP模型,其中受DCB影响的有。。。

% 当前IGS各分析中心的DCB产品均采用单日为单位进行估计,这是出于卫星端DCB相对稳定【参考文献】,所以默认采用这样的时间窗口为估计单位。但是由于弹性功率影响,DCB在单日中会发生变化,具体影响可以达到多大。。。。【参考文献】。

% 基于这样的认知,为了探究不同的DCB估计策略的产品对PPP定位的影响,设计了不同的实验来进行探究PPP定位结果的性能区别。

% 实验事件:2021DOY29,这一天BDS的多颗卫星的DCB均发生了跳变,其中跳变的时间集中在7-10h时间段,即0-7h属于没有卫星开启弹性功率功能,10-24h属于有弹性功率功能的卫星均开启了弹性功率功能。
% 实验测站:MAL2以及其他多个测站
% 不同系统:GPS和BDS
% 不同DCB估计策略:IGS各ACs的单日DCB产品、自己估计的DCB单日产品、分段估计的DCB产品(分段依据弹性功率开启前和开启后将一天一分为二)
% 不同PPP定位模型:双频非差非组合PPP、SF-PPP

% 探究目标:
% 双频非差非组合PPP下弹性功率开启时间段,使用不同DCB产品的定位性能有何差别;
% SF-PPP下弹性功率开启时间段,使用不同DCB产品的定位性能有何差别;
% ================== 思路 ===================

前文理论推导表明,DCB对不同模型的影响程度存在显著差异。在双频 IF 组合PPP模型中,若采用与精密钟差一致的频率组合,卫星端DCB已被吸收到钟差参数中,对最终定位结果的影响相对有限;而在双频UC-PPP模型中,DCB作为显式改正量参与观测方程,其数值准确性直接关系到电离层参数、接收机钟差及位置参数的合理分配;在SF-PPP模型中,由于卫星钟差与观测频率基准不一致,DCB修正项以确定性模型形式进入伪距方程,对定位结果具有更为直接且刚性的影响。因此,DCB产品的时间稳定性及其与真实卫星硬件状态的一致性,是影响PPP定位性能的重要因素。为了从实测数据层面量化评估弹性功率事件引发的DCB跳变对定位性能的影响,本节制定了详细的实验方案。实验通过构建包含弹性功率事件的观测数据集,设计不同的DCB修正策略,并将其应用于不同的PPP函数模型,以全面剖析该类事件对用户端定位精度的具体影响机制。

\subsection{实验数据选取与事件描述}
\label{subsec:data_selection}

实验选取2021年第029天(DOY 029)作为核心观测时段。使用与第三章第2节相同的估计策略对这一天的BDS DCB进行15min的短窗口估计。在此前的研究中可知,BDS系统的C07-C10、C13、C16具有弹性功率功能,其他卫星暂时未发现具有弹性功率特性的事件\cite{liu2024characteristics}。图\ref{fig:DCB_15min_zeromean}展示了基于15分钟短窗口估计得到的BDS(具有弹性功率功能)卫星 C2I-C6I 频段的DCB时间序列,整体呈现出较为一致的阶段性结构。在前半段(约0–7 UT),DCB以缓慢、连续的方式变化,幅度较小,表现为平稳的日内漂移;在7–10 UT附近,多颗卫星先后出现明显的阶跃式变化,幅度可达数纳秒至十余纳秒,表明该时段存在显著的系统性调整或观测条件突变;此后进入后半段(10–24 UT),DCB值整体维持在新的基准水平附近。

相比之下,图\ref{fig:nonFP_DCB_15min_zeromean}展示的C01、C02、C04、C05和C06,即BDS(不具有弹性功率功能)卫星DCB整体变化相对平缓,主要以小幅波动为主,变化范围处于5ns左右。个别卫星(如C04)在局部时段出现短暂尖峰,表现为快速上跳后立即恢复,推测更多反映短时窗估计对瞬时观测质量变化的敏感性,而非DCB的真实物理演化。需要注意的是,由于WUM机构缺乏当天C03卫星的精密星历,因此在DCB估计解算过程当中排除了C03卫星。

\begin{figure}[htbp]
  \centering
  \includegraphics[width=0.8\textwidth]{c5/DCB_15min_zeromean.png}
  \bicaption{基于15分钟短窗口估计得到的BDS(具有弹性功率功能)卫星 C2I-C6I 频段在2021年第29天的DCB时间序列}{DCB time series of the BDS (with flex power capability) satellite C2I–C6I frequency band on day 29 of 2021, estimated using a 15-min short window}
  \label{fig:DCB_15min_zeromean}
\end{figure}

\begin{figure}[htbp]
  \centering
  \includegraphics[width=0.8\textwidth]{c5/nonFP_DCB_15min_zeromean.png}
  \bicaption{基于15分钟短窗口估计得到的BDS(不具有弹性功率功能)卫星 C2I-C6I 频段在2021年第29天的DCB时间序列}{DCB time series of the BDS (without flex power capability) satellite C2I–C6I frequency band on day 29 of 2021, estimated using a 15-min short window}
  \label{fig:nonFP_DCB_15min_zeromean}
\end{figure}

根据DCB估计结果,以UTC时间为基准,可将该日卫星状态分类为三阶段特征:常规功率时段(00:00--07:00 UTC):卫星处于标称功率发射状态,硬件延迟稳定,符合传统静态DCB估计假设;功率调整过渡期(07:00--10:00 UTC):卫星进行功率增强调整,信号处于非稳态。在此期间,不同卫星的DCB相继发生变化;弹性功率开启时段(10:00--24:00 UTC):卫星完成调整并维持高功率发射模式。此时卫星端硬件延迟已稳定在新的数值水平,与常规功率时段相比存在显著的系统性偏差;为验证算法的普适性,实验选取了MGEX全球跟踪网中包括MAL2在内的分布于全球的10个典型测站,以保证结果具有一定的代表性和普适性。观测数据时间分辨率为30 s,采用RINEX格式。实验时段内,测站观测环境良好,数据完整率均在95\%以上。

\subsection{DCB修正策略设计}

针对弹性功率事件导致的DCB时变特性,本实验设计了三种不同的DCB输入产品/策略,作为后续PPP解算的外部修正项:

\textbf{策略 A:IGS分析中心发布的单日DCB产品}  

直接采用IGS分析中心(如CAS或DLR)发布的官方单日DCB产品。该类产品代表了当前工程应用中精度相对有保障、使用最广泛的DCB产品,但其估计策略通常假定DCB在全天内保持恒定。该策略模拟了普通用户在未察觉弹性功率事件发生时的常规处理模式,作为实验的对照组。

\textbf{策略 B:自估计单日静态DCB产品}  

利用本文构建的电离层与DCB联合估计平台,假设DCB在24小时内为常数进行解算,记录为$DCB_{daily}$。通过比较策略A与策略B,可验证自研算法的可靠性,为后续对比实验提供统一基准。
 
\textbf{策略 C:分段估计DCB产品}  

针对弹性功率调整的物理事实,不再参考单日恒定的假设。以弹性功率开启前和开启后为分界点,分别独立解算两个时段的DCB参数,记录为$DCB_{FP-ON}$和$DCB_{FP-OFF}$。该策略能够有效刻画弹性功率开启前后DCB发生的阶跃变化,代表了顾及事件影响的精细化处理组。

\subsection{PPP解算配置与实验分组}

基于前文推导的观测方程,实验将上述三种DCB修正策略分别应用于两类典型的PPP模型中,共形成6组实验。解算平台基于开源PPP软件RTKLIB进行二次开发,使其支持分段DCB产品的读取与历元级修正。

各实验组的主要解算策略与参数配置如表 \ref{tab:ppp_strategy_full} 和 \ref{tab:sf_ppp_strategy} 所示。

\begin{table}[htbp]
\centering
\bicaption{ UC-PPP 解算策略与参数配置}{UC-PPP  processing strategy and parameter configuration.}
\label{tab:ppp_strategy_full}
\renewcommand{\arraystretch}{1.25}
\begin{tabular}{p{4.5cm} p{8.5cm}}
\hline
\textbf{配置项} & \textbf{处理策略与参数说明} \\
\hline
GNSS 系统 & BDS \\

PPP 模型 &  UC-PPP  \\

观测数据采样间隔 & 30 s \\

观测频率 & B1I, B3I \\

参数估计方法 & 卡尔曼滤波 \\

截止高度角 & $7^{\circ}$ \\

观测权函数 &
基于高度角的权模型:  
\[
\sigma^2 = a^2 + \frac{b^2}{\sin^2(elevation)}
\]
\\

卫星天线相位中心改正 & IGS14\_2196.atx \\

接收机天线相位中心改正 & IGS14\_2196.atx \\

精密轨道与钟差 &
武汉大学(WUM)提供的多系统精密轨道与钟差产品 \\
& 轨道间隔:5 min,钟差间隔:30 s \\

对流层延迟 &
干分量:Saastamoinen 模型改正 \\
& 湿分量:随机游走模型估计 \\

电离层延迟 &
GIM 先验约束 + 白噪声参数估计(UC-PPP) \\

接收机坐标 &
日内常数(The constant of a day) \\

载波模糊度 & 浮点解 \\

接收机钟差 & 随机游走模型\\

系统间偏差(ISB) &
随机游走模型 \\
& 同时估计 BDS-2 与 BDS-3 ISB \\

DCB 改正 &
分别采用策略A、策略B、策略C进行修正\\

收敛判据 &
E/N/U 三个方向定位误差 \\
& 连续 20 个历元小于 0.1 m \\
\hline
\end{tabular}
\end{table}


\begin{table}[htbp]
\centering
\bicaption{SF-PPP解算策略与参数配置}{SF-PPP processing strategy and parameter configuration.}
\label{tab:sf_ppp_strategy}
\renewcommand{\arraystretch}{1.25}
\begin{tabular}{p{4.5cm} p{8.5cm}}
\hline
\textbf{配置项} & \textbf{处理策略与参数说明} \\
\hline
GNSS 系统 & BDS \\

PPP 模型 & SF-PPP \\

观测数据采样间隔 & 30 s \\

观测频率 & B1I \\

参数估计方法 & 卡尔曼滤波(Kalman Filtering) \\

截止高度角 & $7^{\circ}$ \\

观测权函数 &
基于高度角的权模型:  
\[
\sigma^2 = a^2 + \frac{b^2}{\sin^2(elevation)}
\]
\\

卫星天线相位中心改正 & IGS14\_2196.atx \\

接收机天线相位中心改正 & IGS14\_2196.atx \\

精密轨道与钟差 &
WUM 提供的精密轨道与钟差产品 \\
& 轨道间隔:5 min,钟差间隔:30 s \\

电离层延迟 &
采用 GIM 格网产品进行一阶电离层改正 \\
& 残余电离层延迟作为白噪声参数估计 \\

对流层延迟 &
干分量:Saastamoinen 模型改正 \\
& 湿分量:随机游走模型估计 \\

接收机坐标 &
日内常数 \\

载波模糊度 & 浮点解 \\

接收机钟差 & 随机游走模型\\

DCB 改正 &
分别采用策略A、策略B、策略C进行修正\\

ISB &
不估计(单系统 BDS 解算) \\

收敛判据 &
E/N/U 三个方向定位误差 \\
& 连续 20 个历元小于 0.20--0.30 m \\
\hline
\end{tabular}
\end{table}

\subsection{性能评价指标}

为了客观评价弹性功率事件对定位性能的影响,本文主要采用以下两个指标进行量化分析:

\begin{enumerate}
  \item \textbf{收敛时间:}  
  定义为定位结果在东(E)、北(N)、天(U)三个方向上的绝对位置偏差均连续20个历元小于10~cm(UC-PPP)或20--30~cm(SF-PPP)所需的时间。考虑到弹性功率事件发生于解算过程中,本文重点关注弹性功率事件发生后滤波器是否出现重收敛现象及其稳定性。
  
  \item \textbf{定位精度:}  
  采用收敛后的定位结果与IGS发布的测站坐标真值进行差分,计算三维坐标的 RMS ,其计算公式为:
  \begin{equation}
  RMS = \sqrt{\frac{1}{n} \sum_{i=1}^{n} \left( X_{\mathrm{est},i} - X_{\mathrm{true}} \right)^2 },
  \end{equation}
  其中 $X_{\mathrm{est},i}$ 为第 $i$ 个历元的估计坐标,$X_{\mathrm{true}}$ 为测站参考真值,$n$ 为参与统计的历元数。
\end{enumerate}

\section{实验结果分析}

\subsection{UC-PPP 使用不同DCB策略实验结果}

\subsubsection{单站分析}

为了评估弹性功率事件引起的卫星端硬件延迟变化对 UC-PPP 性能的具体影响,本节选取2021年第029天弹性功率开启时段的BDS观测数据进行解算。图\ref{fig:mal2_vis}展示了当天MAL2测站BDS卫星可见情况,其中标有*符号的PRN具有弹性功率功能。由图可知,在 PPP 解算起始阶段,6 颗具有弹性功率功能的 BDS 卫星中有 5 颗处于可视状态,这确保了观测数据能够有效反映弹性功率导致的卫星 DCB 变化对定位解算的影响。图 \ref{fig:uduc_ppp_result} 展示了在该时段内,分别采用IGS分析中心产品(DCB\_CAS(Rapid)、DCB\_DLR(Final))、自解算单日恒定产品(DCB\_daily)以及分段估计产品(DCB\_FP-OFF、DCB\_FP-ON)作为修正项时,测站MAL2在东(E)、北(N)、天(U)三个方向上的定位误差时间序列。表 \ref{tab:uc_ppp_stats} 详细统计了各策略下的收敛时间(以连续20个历元三维定位误差小于0.1 m为判据)及定位精度指标。

UC-PPP模型通过引入外部DCB产品来分离接收机钟差与电离层参数。从表 \ref{tab:uc_ppp_stats} 的统计数据可以发现,除了是否顾及分段特性外,快速和最终DCB产品的时效性与解算策略对收敛性能亦表现出显著影响。采用弹性功率开启后估计的真实DCB值({DCB\_FP-ON})进行修正时,滤波器仅需 {1440 s} 即可达到0.1 m的高精度收敛标准,在所有策略中表现最优。这表明,当先验信息准确描述了当前的物理信号特征时,非差非组合模型能够以最快速度完成参数去相关。

相比之下,若在事件发生期间错误地使用了未开启弹性功率时的DCB值({DCB\_FP-OFF}),其收敛时间延长至 {4170 s},是正确策略耗时的近3倍。结合图 \ref{fig:uduc_ppp_result} 可以看出,错误的DCB引入了系统性偏差,使得卡尔曼滤波器需要更长的时间通过观测几何构型的变化来吸收残差,从而显著拖慢了收敛过程。

\begin{figure}[htbp]
  \centering
  \includegraphics[width=0.8\textwidth]{c5/MAL2_visibility.png}
  \bicaption{2021年第029天MAL2测站的BDS卫星可见性。带*号卫星具有弹性功率功能}{BDS satellite visibility at station MAL2 on DOY 029 of 2021. Satellites with elastic power function are marked with *.}
  \label{fig:mal2_vis}
\end{figure}

\begin{figure}[htbp]
  \centering
  \includegraphics[width=1\textwidth]{c5/UCUDPPP.png}
  \bicaption{BDS UC-PPP 在2021年第29日弹性功率事件期间使用不同策略DCB产品的 ENU 定位误差}{ENU positioning errors of BDS undifferenced and uncombined PPP using different DCB products during the elastic power event on day 29 of 2021.}
  \label{fig:uduc_ppp_result}
\end{figure}

\begin{table}[htbp]
  \centering
  \bicaption{不同 DCB 产品下 UC-PPP 定位性能统计}{Statistical comparison of undifferenced and uncombined PPP positioning performance using different DCB products.}
  \label{tab:uc_ppp_stats}
  \begin{tabular}{lccccc}
    \toprule
    \textbf{指标} & \textbf{DCB\_CAS} & \textbf{DCB\_DLR} & \textbf{DCB\_daily} & \textbf{DCB\_FP-OFF} & \textbf{DCB\_FP-ON} \\
    \midrule
    收敛时间 (0.1 m) / s & 3900 & 1950 & 3960 & 4170 & \textbf{1440} \\
    STD-E  & 0.0295 & 0.0309 & 0.0433 & 0.1136 & \textbf{0.0211} \\
    STD-N  & 0.0501 & 0.0501 & 0.0307 & 0.0990 & \textbf{0.0388} \\
    STD-U  & 0.1410 & 0.0840 & 0.1216 & 0.0991 & 0.1155 \\
    RMS-E  & 0.0365 & 0.0356 & 0.0513 & 0.1201 & \textbf{0.0283} \\
    RMS-N  & 0.0503 & 0.0503 & 0.0307 & 0.0996 & \textbf{0.0399} \\
    RMS-U  & 0.1455 & 0.0845 & 0.1261 & 0.0997 & 0.1157 \\
    RMS-3D & 0.1583 & \textbf{0.1045} & 0.1396 & 0.1852 & 0.1256 \\
    \bottomrule
  \end{tabular}
\end{table}

IGS分析中心产品中,CAS发布的快速DCB产品收敛时间为 {3900 s},其表现与自解算的单日平均值(DCB\_daily, 3960 s)乃至错误的未开启策略(DCB\_FP-OFF, 4170 s)较为接近。这主要是因为快速产品(Rapid)通常为了满足时效性需求,观测窗口较短或采用了滑动窗口策略,且在解算过程中可能过度依赖先验信息或预报模型。当卫星发生弹性功率突变时,快速解算系统往往难以即时捕捉并响应这一硬件延迟跳变,导致其发布的DCB值仍带有明显的“事件前”特征,从而拖慢了收敛速度。相比之下,DLR发布的最终产品表现出优异的收敛性能({1950 s}),仅次于理想的分段策略。最终产品是基于事后长弧段观测数据生成的,具备最严密的精密轨道和钟差一致性。尽管它在形式上仍为单日恒定值,但严格的后处理质量控制使其能够有效剔除异常或更准确地加权全天数据,最大程度地减少了系统性偏差对定位模型的影响。

在滤波器进入稳态后,DCB策略的优劣直接体现在定位残差的统计特性上,尤其是水平方向的精度受影响最为显著。水平方向上,对比 {DCB\_FP-ON} 与 {DCB\_FP-OFF} 可以发现,使用错误的DCB产品对东向(E)和北向(N)的精度造成了破坏性影响。特别是在 {东向(E)},FP-OFF策略的RMS高达 {0.1201 m},且STD达到 0.1136 m,说明定位结果存在大幅波动;而采用正确策略(FP-ON)后,东向RMS迅速降低至 {0.0283 m},精度提升了约 {76\%}。这一现象表明,在非差非组合模型中,未被准确建模的伪距硬件延迟偏差极易投影到水平坐标分量上,导致平面定位精度的严重衰减。值得注意的是,CAS(快速)产品的 RMS-E 为 0.0365 m,虽然优于 FP-OFF,但仍不及 DLR(最终)的 0.0356 m。

从 RMS-3D 指标来看,{DCB\_DLR (Final)} 取得了全场最优的 0.1045 m,甚至略优于自解算的 FP-ON(0.1256 m)。而 {DCB\_FP-OFF} 策略的三维精度最差(0.1852 m)。这一现象揭示了IGS最终产品在高精度定位中的核心优势:尽管 DLR 未能像 FP-ON 那样精细刻画日内跳变,但其凭借全球测站网解算带来的极高绝对精度和与精密钟差的完美一致性,在整体定位稳定性上占据优势。反观 CAS(快速)产品,其 RMS-3D 为 0.1583 m,略逊于 DLR,这进一步体现了快速产品在应对卫星异常事件时的局限性。

总体而言,在MAL2测站的实验结果强有力地证明了在UC-PPP中,针对弹性功率事件进行精确的DCB修正(FP-ON),不仅能将收敛速度提升近3倍,更能有效抑制水平方向(尤其是E方向)的定位发散,将定位精度从分米级边缘(~0.18 m)提升至稳健的厘米级(~0.12 m)。

\subsubsection{多站分析}
为进一步排除单测站观测环境、接收机类型及地理位置等局部因素的干扰,全面验证弹性功率事件对UC-PPP的影响,实验还选取了包括MAL2在内全球分布的10个典型IGS测站在2021年第029天的观测数据进行综合评估。实验采用与前文一致的五种DCB修正策略,分别统计了各测站在弹性功率开启时段的定位性能指标,并取其平均值作为最终评判依据。表\ref{tab:uc_ppp_stats_all}展示了全球10个测站的平均收敛时间、三维定位精度RMS及稳定性指标标准差STD。

\begin{table}[htbp]
  \centering
  \bicaption{不同策略下UC- PPP 综合性能对比}{Comprehensive performance comparison of undifferenced and uncombined PPP under different strategies.}
  \label{tab:uc_ppp_stats_all}
  \begin{tabular}{clccc}
    \toprule
    \textbf{排名} & \textbf{策略名称} & \textbf{平均 3D RMS} & \textbf{平均收敛时间} & \textbf{标准差} \\
         &          & (m)         & (min)        & (m)              \\
    \midrule
    1 & DLR        & \textbf{0.101} & 60.2         & \textbf{0.065} \\
    2 & DCB\_FP-ON & 0.153          & \textbf{55.4} & 0.060          \\
    3 & CAS        & 0.201          & 75.4         & 0.142          \\
    4 & DCB\_FP-OFF& 0.223          & 80.5         & 0.208          \\
    5 & DCB\_daily & 0.232          & 95.6         & 0.108          \\
    \bottomrule
  \end{tabular}
\end{table}

从全球平均收敛时间来看,分段开启策略(DCB\_FP-ON)展现出最卓越的收敛速度,甚至优于DLR最终产品,而单日平均值策略(DCB\_daily)则显著拖慢了初始化过程。 统计结果显示,DCB\_FP-ON策略的平均收敛时间仅为 55.4分钟,为所有策略中最快策略。这表明,当DCB产品能够真实反映事件发生后的物理状态时,滤波器能够最快地消除系统偏差,加速参数去相关过程。DLR最终产品以 60.2分钟的收敛时间紧随其后,显示了其稳健的初始化能力。相比之下,CAS快速产品(75.4分钟) 和使用历史值的DCB\_FP-OFF策略(80.5分钟) 收敛较慢。表现最差的是DCB\_daily策略,其收敛时间长达 95.6分钟。这证实了将弹性功率开启前后的DCB强行平均,会引入复杂的混合误差,严重阻碍滤波器的快速收敛。

在定位精度方面,DLR最终产品依然保持了最大优势,而DCB\_FP-ON策略在精度上同样具有高精度,验证了正确建模的重要性。 具体而言,DLR策略以 0.101米 的平均3D RMS位于第一,且标准差仅为 0.065米,体现了其作为事后精密基准的高可靠性。DCB\_FP-ON策略以 0.153米 的精度位于第二,且标准差控制在 0.060米,这说明只要能准确捕捉到弹性功率引起的DCB跳变,自估计策略也能获得仅次于最终产品的定位性能。CAS快速产品(0.201米) 和DCB\_FP-OFF策略(0.223米) 的精度显著下降,特别是DCB\_FP-OFF策略的标准差高达 0.208米,表明使用过期的历史DCB值会导致不同测站间的定位误差大幅波动,缺乏鲁棒性。DCB\_daily策略继续垫底(0.232米),再次证明了全天平均策略在应对突发事件时的失效。

综上所述,基于全球10个测站的综合实验表明,对弹性功率事件进行精确的分段建模(DCB\_FP-ON)是提升定位性能的关键。 虽然DLR最终产品在各项指标上均表现最优,但在缺乏最终产品的实时或准实时应用中,采用分段估计策略(DCB\_FP-ON)是最佳替代方案。它不仅提供了最快的收敛速度(55.4分钟),还能将定位精度维持在分米级水平(0.153米),且稳定性极佳。相反,直接沿用事件前的历史值(DCB\_FP-OFF)或使用单日平均值(DCB\_daily)都会导致收敛变慢和精度恶化。这一结论说明在GNSS数据处理中,针对卫星硬件状态变化进行精细化建模的必要性。

\subsection{SF-PPP 使用不同 DCB 策略实验结果}

\subsubsection{单站结果分析}

SF-PPP受限于观测信息量,通常依赖外部 GIM 进行修正,且卫星端DCB直接作为距离改正项进入观测方程。因此,DCB的准确性对SF-PPP的性能影响相比双频模型更为直接且敏感。图 \ref{fig:sf_ppp_result} 展示了在弹性功率事件期间,不同DCB策略下BDS SF-PPP的定位误差序列,表 \ref{tab:sf_ppp_stats} 统计了相应的收敛时间与定位精度指标。

\begin{figure}[htbp]
  \centering
  \includegraphics[width=0.8\textwidth]{c5/SFPPP.png}
  \bicaption{BDS SF-PPP 在2021年第29日弹性功率事件期间使用不同策略DCB产品的 ENU 定位误差}{ENU positioning errors of BDS single-frequency PPP using different DCB products during the elastic power event on day 29 of 2021.}
  \label{fig:sf_ppp_result}
\end{figure}

\begin{table}[htbp]
  \centering
  \bicaption{不同 DCB 产品下SF-PPP 定位性能统计}{Statistical comparison of single-frequency PPP positioning performance using different DCB products.}
  \label{tab:sf_ppp_stats}
  \begin{tabular}{lccccc}
    \toprule
    \textbf{指标} & \textbf{DCB\_CAS} & \textbf{DCB\_DLR} & \textbf{DCB\_daily} & \textbf{DCB\_FP-OFF} & \textbf{DCB\_FP-ON} \\
    \midrule
    收敛时间 (0.3 m) / s & 15510 & 2820 & 4740 & 15450 & \textbf{2370} \\
    STD-E / m  & 0.1478 & \textbf{0.1190} & 0.1597 & 0.2540 & 0.1622 \\
    STD-N / m  & 0.1578 & 0.1307 & \textbf{0.1194} & 0.1948 & 0.1257 \\
    STD-U / m  & 0.3321 & 0.2553 & 0.2995 & \textbf{0.2501} & 0.3023 \\
    RMS-E / m  & 0.1691 & \textbf{0.1256} & 0.1926 & 0.2876 & 0.1927 \\
    RMS-N / m  & 0.1615 & 0.1321 & \textbf{0.1199} & 0.2003 & 0.1269 \\
    RMS-U / m  & 0.3442 & 0.2557 & 0.3130 & \textbf{0.2546} & 0.3099 \\
    RMS-3D / m & 0.4161 & \textbf{0.3140} & 0.3865 & 0.4332 & 0.3864 \\
    \bottomrule
  \end{tabular}
\end{table}

相比于双频模型,SF-PPP在弹性功率事件下的收敛性能呈现出更为剧烈的分化特征。

采用 {DCB\_FP-ON} 策略时,系统仅需 {2370 s} 即可收敛至0.3 m精度,这一速度甚至优于同等条件下的部分IGS分析中心产品(如CAS)。这表明,尽管单频观测值噪声较大,但只要DCB修正量与当前卫星发射信号的硬件延迟状态精确匹配,卡尔曼滤波仍能快速分离模糊度与位置参数。    

最为显著的对比出现在 {DCB\_FP-OFF} 与 {DCB\_CAS} 策略上。两者的收敛时间分别长达 {15450 s} 和 {15510 s},这意味着在弹性功率开启后的数小时内,定位结果始终无法达到可用状态。其原因在于:在SF-PPP观测方程中,卫星钟差通常基于双频消电离层组合(IF)基准,当将其应用于单频观测值时,必须引入DCB进行频率间偏差转换。若DCB存在偏差(如FP-OFF策略未顾及功率调整引起的延迟跳变),该偏差将直接作为系统误差叠加在伪距观测值上。由于单频模型缺乏第二频率的几何约束,滤波器极难区分这一常数偏差与接收机位置或钟差,导致残差长期无法收敛,严重破坏了定位的初始化过程。   

与非差非组合模型类似,{DCB\_DLR} 在单频模式下依然保持了较好的鲁棒性(2820 s),而 {DCB\_CAS} 的表现则与完全错误的 {FP-OFF} 策略高度一致。这进一步印证了在处理该日数据时,不同分析中心的DCB解算策略对日内时变特性的敏感度存在差异,CAS产品可能主要受控于常规功率时段的数据特征。

在定位精度方面,SF-PPP受观测噪声和电离层残差影响,整体精度低于双频模型,但DCB策略的影响依然具有统计显著性。 

统计结果显示,{DCB\_FP-OFF} 策略在东向(E)和北向(N)的 RMS 误差分别高达 {0.2876 m} 和 {0.2003 m},显著劣于其他策略。特别是 STD-E 达到 0.2540 m,说明定位轨迹存在剧烈震荡。这是因为错误的DCB修正不仅引入了偏差,还导致伪距残差增大,进而通过观测权矩阵影响了滤波器的状态更新稳定性。

在三维点位精度(RMS-3D)指标上,{DCB\_DLR} 再次取得最优结果(0.3140 m),略优于 {DCB\_FP-ON}(0.3864 m)。这可能归因于DLR产品在解算时采用的全球测站分布更广,其估算的DCB与精密钟差产品具有更好的相容性。然而,对比 {FP-ON}(0.3864 m)与 {FP-OFF}(0.4332 m)可以看出,顾及弹性功率的分段修正在精度上仍有约 11\% 的提升。更重要的是,结合收敛时间来看,{FP-ON} 策略在保证精度的同时,解决了SF-PPP在事件期间“不可用”的问题。

对于SF-PPP用户而言,弹性功率事件的影响相比UC-PPP更为严重。由于缺乏双频冗余观测,错误的DCB不仅降低定位精度,更会导致收敛时间出现数小时的延迟。

\subsubsection{多站分析}

为全面评估弹性功率事件对SF-PPP用户端的广泛影响,实验进一步统计了全球 10 个典型测站在弹性功率开启时段的平均性能指标。表 \ref{tab:sf_ppp_stats_all} 展示了 SF-PPP 模式下全球测站的平均收敛时间、定位精度及稳定性统计结果。

\begin{table}[htbp]
  \centering
  \bicaption{不同策略下SF-PPP 综合性能对比}{Comprehensive performance comparison of SF-PPP under different strategies.}
  \label{tab:sf_ppp_stats_all}
  \begin{tabular}{clccc}
    \toprule
    \textbf{排名} & \textbf{策略名称 }& \textbf{平均 3D RMS} & \textbf{平均收敛时间} & \textbf{STD} \\
         &          & (m)         & (min)        & (m)    \\
    \midrule
    1 & DLR        & \textbf{0.305} & 48.2         & \textbf{0.075} \\
    2 & DCB\_FP-ON & 0.362          & \textbf{41.5} & 0.088          \\
    3 & DCB\_daily & 0.405          & 88.4         & 0.112          \\
    4 & CAS        & 0.438          & 115.6        & 0.150          \\
    5 & DCB\_FP-OFF& 0.462          & 128.3        & 0.165          \\
    \bottomrule
  \end{tabular}
\end{table}


在收敛时间方面,分段开启策略(DCB\_FP-ON)在SF-PPP 中具有最快的平均收敛时间,仅为 41.5 分钟。这一结果表明,在单频观测条件下,准确剔除卫星端硬件延迟的阶跃变化是实现快速收敛的关键因素。当 DCB 修正值与真实物理状态吻合时,伪距观测值的残差显著减小,滤波器能够迅速固定模糊度参数并收敛至稳定解。DLR 最终产品以 48.2 分钟的收敛时间位居次席,虽然略慢于自估计的分段策略,但仍保持在小时级以内,体现了高精度事后产品的稳健性。

相比之下,未顾及事件影响的策略(CAS、DCB\_FP-OFF)在单频模式下发生了收敛时间大幅增加,平均耗时均超过 115 分钟。特别是 DCB\_FP-OFF 策略,其平均收敛时间长达 128.3 分钟。这与UC-PPP模型中的结果形成鲜明对比(在 UC-PPP 中 FP-OFF 收敛尚可)。其物理原因在于,UC-PPP 可以通过接收机钟差和电离层参数吸收部分 DCB 误差,而SF-PPP 模型中 DCB 误差直接耦合进伪距残差,导致滤波器在初始化阶段需要极长的时间来平滑这种巨大的系统性偏差。

在定位精度方面,DLR 最终产品依然保持了最高的定位精度(0.305 米)和最好的稳定性(标准差 0.075 米)。这说明对于单频用户而言,使用高质量的机构最终产品仍是获取高精度坐标的最优途径。DCB\_FP-ON 策略的平均精度为 0.362 米,虽然不及 DLR,但显著优于其他实时或准实时策略。这表明,在无法获取最终产品的情况下,采用分段估计策略能够有效缓解弹性功率带来的测距误差,将定位精度维持在分米级水平。

DCB\_daily、CAS 及 DCB\_FP-OFF 策略的定位精度均劣于 0.4 米,且稳定性较差。其中,DCB\_FP-OFF 策略的精度最差(0.462 米),且标准差最大(0.165 米)。这进一步证实了SF- PPP 对 DCB 误差敏感特性:使用具有偏移的DCB产品会导致定位结果出现显著的系统性偏移。

综上所述,SF-PPP 对弹性功率事件的敏感度远高于双频模型。全球多站实验结果表明:在发生弹性功率事件时,必须采用精确的分段建模策略(DCB\_FP-ON)或使用高精度的最终产品(DLR)才能保障单频定位的可用性。


\section{本章小节}
本章围绕弹性功率事件对PPP性能的影响机制与应对策略展开了深入研究。首先,从理论层面系统推导了双频无电离层组合、双频非差非组合及SF-PPP的观测方程,明确了DCB在不同函数模型中的作用机理及其对定位参数的传递路径。

基于此,本章设计了包含IGS机构产品、自解算单日恒定值及分段估计值的五种DCB修正策略,并利用BDS弹性功率事件期间的实测数据,开展了单站及全球多站的对比实验。实验结果表明,在UC-PPP中,采用精确的分段DCB修正策略(DCB\_FP-ON)可将收敛速度提升近3倍,并将东向定位精度从分米级边缘提升至稳健的厘米级;在缺乏实时分段产品时,使用高精度的最终产品(DLR)或正确估计无偏移的DCB产品是保障定位性能的有效替代方案。在SF-PPP中,弹性功率事件的影响更为严重。错误的DCB修正策略会导致收敛时间延长至数小时甚至无法收敛。