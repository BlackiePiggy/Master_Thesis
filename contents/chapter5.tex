% !TEX root = ../main.tex

\chapter{星载产品多维时序异常检测方法}

\section{基于深度学习的多维时序检测}

\subsection{背景,现有方法和缺陷}

目前,针对星载产品在轨运行状态的健康监测与异常检测,国内外主要采用以下几类技术路线,这些技术方案与本发明所关注的问题最为接近

\subsection{基于时序自编码器的多维遥测异常检测方法}

针对星载产品多通道遥测数据维度高、时间跨度长且异常样本稀缺的问题,本文提出一种基于时序自编码器的多维时序异常检测方法。该方法在保证时间连续性的前提下,对遥测数据进行分段建模与多尺度特征提取,并通过无监督时序自编码器学习正常行为模式,最终基于重构误差实现异常检测。方法整体流程如图\ref{fig:pat1_pic1}所示,主要包括数据预处理、时间分段、特征构建、时序建模以及异常判定五个步骤。

\begin{figure}[htbp]
\centering
\includegraphics[width=0.3\linewidth]{figures/patent1/pic1.png}
\caption{基于深度学习的星载产品分段遥测时序异常检测方法流程图}
\label{fig:pat1_pic1}
\end{figure}

\begin{figure}[htbp]
\centering
\includegraphics[width=1\linewidth]{figures/patent1/pic2.png}
\caption{数据预处理与特征矩阵构建方法}
\label{fig:pat1_pic2}
\end{figure}

\begin{figure}[htbp]
\centering
\includegraphics[width=1\linewidth]{figures/patent1/pic3.png}
\caption{时序自编码器模型设计与LSTM单元结构}
\label{fig:pat1_pic3}
\end{figure}

首先,对星载产品下行的多通道遥测数据进行统一获取与质量预处理,剔除误码帧、异常采样帧以及非业务采样数据,仅保留采样时间间隔规律的有效遥测序列。设质量过滤后的多通道遥测数据表示为
\begin{equation}
\mathbf{X}_{\mathrm{valid}}
=
\left\{
\left(
x_t^{(1)}, x_t^{(2)}, \ldots, x_t^{(N_c)}
\right)
\right\}_{t=1}^{T}
\end{equation}
其中,$x_t^{(c)}$ 表示第 $c$ 个遥测通道在时刻 $t$ 的有效观测值。

为避免后续处理跨越非连续时间区间,依据相邻时间戳间隔对遥测序列进行自动分段。通过计算相邻时间差 $\Delta t_i = t_{i+1}-t_i$,当其超过阈值 $T_{\mathrm{gap}}$ 时识别为分段边界,从而将完整遥测序列划分为若干连续数据段 $S_j$。后续所有滑动窗口、统计特征计算以及序列构建操作均严格限制在单一数据段内完成,以避免时间不连续性对建模结果的干扰。

在时间分段约束下,对各遥测通道进行多尺度平滑去噪与趋势提取。如图\ref{fig:pat1_pic2}所示,设定一组平滑窗口尺度,对原始信号计算滑动平均以抑制高频噪声,并通过信噪比增益指标选择各通道的主平滑尺度。在此基础上,引入多窗口滚动统计方法,从平滑信号中提取滚动均值、标准差、极值、变化范围以及一阶变化率等统计特征。对于任一时刻 $k$,将所有通道的统计特征进行拼接,构成整体特征向量 $\mathbf{f}_k$,从而得到特征矩阵
\begin{equation}
\mathbf{X}_{\mathrm{features}}
=
\left[
\mathbf{f}_1, \mathbf{f}_2, \ldots, \mathbf{f}_K
\right]^{\mathrm{T}}
\in \mathbb{R}^{K \times D_{\mathrm{total}}}
\end{equation}
随后基于训练数据计算特征均值与标准差,对特征进行标准化处理,以保证不同特征维度在同一量纲下参与建模。

如图\ref{fig:pat1_pic3}所示,在完成特征构建与标准化后,引入基于循环神经网络的时序自编码器对多维特征序列进行无监督建模。在每个数据段内,采用滑动窗口方式构造长度为 $L_{\mathrm{seq}}$ 的输入序列:
\begin{equation}
\mathbf{X}_{\mathrm{seq}}
=
\left[
\mathbf{f}_i^{\mathrm{norm}},
\mathbf{f}_{i+1}^{\mathrm{norm}},
\ldots,
\mathbf{f}_{i+L_{\mathrm{seq}}-1}^{\mathrm{norm}}
\right]
\end{equation}
并约束序列不跨越段边界。序列样本输入时序自编码器,经编码器映射至低维潜在空间,再由解码器重构得到输出序列,其映射关系表示为
\begin{equation}
\mathbf{X}_{\mathrm{seq}}
\;\xrightarrow{\;\mathcal{E}_{\theta}\;}\;
\mathbf{z}
\;\xrightarrow{\;\mathcal{D}_{\phi}\;}\;
\hat{\mathbf{X}}_{\mathrm{seq}}
\end{equation}
模型通过最小化输入序列与重构序列之间的均方误差进行训练,从而学习系统在正常工况下的时序演化特征。

在测试阶段,针对每个时间点构造以该点为终点的输入序列,并计算对应的重构误差。由重构结果中提取时间点 $k$ 的重构特征向量 $\hat{\mathbf{f}}_k^{\mathrm{norm}}$,定义逐点重构误差为
\begin{equation}
e_k
=
\left\|
\mathbf{f}_k^{\mathrm{norm}}
-
\hat{\mathbf{f}}_k^{\mathrm{norm}}
\right\|_2^2
\end{equation}
为提高异常评分的鲁棒性,对重构误差序列进行滑动平均平滑,得到异常评分序列 $\{s_k\}$。

最后,基于训练数据对应的异常评分分布构建正常参考分布,并采用百分位数法或固定阈值法确定异常判定阈值 $\tau$。当异常评分超过阈值时判定为异常;同时在数据段起始与结束位置引入边界排除机制,以减少由窗口不完整引起的伪异常,从而获得最终稳定可靠的异常检测结果。


\subsection{实验-基于模拟仿真遥测数据的异常检测实验与效果验证}

\subsubsection{数据集与实验设计}

为验证本发明所提方法的有效性,采用来自中地球轨道卫星的真实星载产品遥测数据进行了测试与分析。实验数据按照真实在轨遥测数据的正常与退化特征生成了时间跨度为一年的模拟数据。数据呈现为多个不等长的数据段,充分体现了星载遥测数据分段、数据缺失不规律的典型特征,与背景技术中所述难点一致.

数据中包含了模拟注入的多种典型异常模式,用于系统性地评估本方法的检测能力:

\begin{itemize}
    \item 斜率变化异常:模拟星载产品性能发生渐进性退化的早期信号。
    \item 离群点异常:模拟因瞬时干扰或数据跳变产生的突发性异常。
    \item 整体漂移异常:模拟因器件老化或环境因素导致的系统性偏差。
    \item 组合异常:由上述三种异常组合构成,用于测试方法在复杂场景下的综合性能。
\end{itemize}

关键参数设置如下:多尺度平滑窗口集合 $\mathcal{W}_{\mathrm{smooth}} = \{50, 100, 200\}$
,以捕捉不同时间尺度的特征变化。基于验证集性能,异常判定固定阈值设定为 4.0。

\subsubsection{实验结果与分析}

通过对测试数据的全面评估,本发明方法对不同类型异常均表现出优异的检测性能,具体分析如下。

\begin{figure}[htbp]
\centering
\includegraphics[width=1\linewidth]{figures/patent1/pic4.png}
\caption{训练数据、测试数据。其中训练数据为模拟星载产品遥测时序数据(左上),测试数据包含正常数据(左中)、斜率漂移异常数据(左下)、离群点异常数据(右上)、整体漂移异常数据(右中)、混合异常数据(右下)}
\label{fig:pat1_pic4}
\end{figure}

\begin{figure}[htbp]
\centering
\includegraphics[width=1\linewidth]{figures/patent1/pic5.png}
\caption{正常数据(左)与斜率漂移异常数据(右)的原始遥测时序、平滑后时序与异常得分时序}
\label{fig:pat1_pic5}
\end{figure}

\begin{figure}[htbp]
\centering
\includegraphics[width=1\linewidth]{figures/patent1/pic6.png}
\caption{离群点异常数据(左)与整体漂移异常数据(右)的原始遥测时序、平滑后时序与异常得分时序}
\label{fig:pat1_pic6}
\end{figure}

\begin{figure}[htbp]
\centering
\includegraphics[width=0.5\linewidth]{figures/patent1/pic7.png}
\caption{混合异常数据的原始遥测时序、平滑后时序与异常得分时序}
\label{fig:pat1_pic7}
\end{figure}

通过对测试数据的全面评估,本发明方法对正常状态与不同类型异常均表现出优异的区分与检测性能。首先,在正常状态测试数据上的评估结果(如图\ref{fig:pat1_pic5}左半部分所示)表明:当原始遥测数据呈现符合预期的正常缓变趋势(如按固有斜率下降)时,本发明方法输出的异常分数始终稳定在较低水平,其均值约为0.5,最大值不超过1.5,远低于预设的异常判定阈值,且仅在有限范围内小幅波动。这充分证明了本发明方法对正常行为模式具有准确的认知,能有效避免将正常的长期趋势误判为异常,从而确保了极低的误报率。

如图\ref{fig:pat1_pic5}右半部分所示,针对斜率变化异常的检测,该方法能够有效捕捉到缓慢的斜率变化。实验表明,在斜率开始变化后约100个数据点,模型输出的异常分数开始显著上升并超过阈值。此有限的延迟是由于模型需要一定的观测窗口以确认趋势的持续性,该延迟处于“亚健康”状态的早期阶段,对于实现故障预警而言是完全可接受的。更重要的是,实验观察到异常分数随异常持续时间的增加而单调递增,这与系统性能退化的物理机理高度吻合,证明了本方法在趋势性异常早期识别方面的有效性。

如图\ref{fig:pat1_pic6}左半部分所示,针对离群点异常的检测,对于所有注入的离群点,本方法均能实现即时、精准的识别。一个值得注意的现象是,在持续时间较长的离群段中部,异常分数会回落至较低水平。此现象揭示了离群点与漂移异常在模型视角下的内在联系:持续的离群段可被视为一种小幅度的漂移,模型将其学习为一种新的“准正常”模式,从而体现了模型对异常语义的理解深度,而非简单的噪声响应。

如图\ref{fig:pat1_pic6}右半部分所示针对整体漂移异常的检测,本方法对整体漂移异常表现出极高的灵敏度,所有处于漂移状态的数据点均被成功标识为异常。异常分数在漂移发生的跳变点处达到峰值(接近160),并在后续的持续漂移阶段稳定在10-20的区间内,清晰地将异常状态与正常背景分离。

如图\ref{fig:pat1_pic7}所示针对组合异常的检测,在包含5处离群点与1处斜率变化的复合异常场景中,本方法除在斜率变化最初始的、最不显著的阶段存在漏报外,对其余所有异常均实现了100\%的检出。特别是在离群点与斜率变化叠加的区域,异常分数呈现出协同增强效应,产生了更高的异常评分,这证明了本方法提供的异常分数能够有效反映异常的综合严重程度。

为量化评估本方法的检测性能,在包含多种注入异常的大规模测试集上进行了统计验证。实验结果表明,本发明方法在异常检测任务中同时实现了高查全率与高精确率。具体而言,其对真实异常的查全率超过97\%,这证明该方法能够极为全面地捕捉到绝大部分的异常模式;同时,其检测结果的精确率高达约99.8\%,这反映了方法具有极低的误报率,能有效保障预警系统的可靠性。这两项核心指标共同证实,本发明在应对星载产品复杂的遥测数据时,能够在近乎排除虚警的前提下,实现对隐性及显性异常的高效、精准识别。

\section{基于时序分解的深度学习异常检测}

\subsection{STL时序分解算法}
STL (Seasonal-Trend decomposition using Loess) 是一种由Cleveland等人提出的、基于局部加权回归 (Locally Weighted Scatterplot Smoothing, LOESS) 的时间序列分解方法\cite{cleveland1990stl}。与传统的基于移动平均或参数模型(如 X-11、SEATS)的分解技术不同,STL 方法具有高度的灵活性和鲁棒性,能够有效应对非线性趋势和复杂的季节性变化。

理论上,任一时间序列 $Y_t$ 都可以通过STL加性分解为三个分量:趋势分量 (Trend)、季节分量 (Seasonal) 和残差分量 (Residual)\cite{peixeiro2022time}。其数学表达如下:
\begin{equation}
    Y_t = T_t + S_t + R_t, \quad t = 1, \dots, N
    \label{eq:stl_decomposition}
\end{equation}
其中:
\begin{itemize}
    \item $Y_t$ 表示在时刻 $t$ 的观测值;
    \item $T_t$ 代表趋势分量;
    \item $S_t$ 代表季节分量;
    \item $R_t$ 为去除趋势和季节后的残差项。
\end{itemize}

STL 的核心思想是通过嵌套的内循环 (Inner Loop) 和外循环 (Outer Loop) 迭代计算各分量。内循环主要负责趋势项和季节项的拟合。在每一次迭代中,算法首先对季节子序列进行 LOESS 平滑以估计季节分量 $S_t$;随后,通过对去季节后的序列 $Y_t - S_t$ 应用 LOESS 平滑来估计趋势分量 $T_t$。外循环 (Outer Loop)主要用于增强算法对异常值的鲁棒性。算法根据内循环产生的残差项 $R_t$ 计算鲁棒性权重 $\rho_t$。对于残差较大的观测点(即潜在的异常值),赋予较小的权重,从而在下一轮内循环的平滑过程中降低其对趋势和季节估计的影响。这种双循环机制使得 STL 不仅能够捕捉随时间动态变化的季节性模式,还能有效抵抗极端值的干扰。

在本研究中采用 STL 方法主要基于以下优势:
\begin{itemize}
    \item \textbf{动态季节性处理能力}:STL 允许季节分量随时间发生缓慢演变,而非强制假设季节性是固定不变的周期函数,这对于长跨度的时间序列分析尤为重要。
    \item \textbf{鲁棒性 (Robustness)}:通过外循环的抗差权重机制,STL 能够有效防止数据中的瞬时突变或测量误差扭曲趋势项的估计。
    \item \textbf{参数灵活性}:研究者可以根据数据的具体特性(如月度或季度数据),通过调整季节窗口 (seasonal window) 和趋势窗口 (trend window) 的平滑参数,精准控制各分量的平滑程度。
\end{itemize}


\subsection{算法流程}

针对星载产品遥测数据维度高、时间跨度长、异常样本极度稀缺等特点,传统依赖固定阈值或单变量统计特征的方法往往只能识别幅度显著的显性异常,难以及时发现隐性的、早期的性能退化问题。为此,提出了一种基于深度学习的星载产品分段多维遥测时序异常检测方法。该方法通过引入分段管理、多尺度去噪、趋势–周期–残差解耦建模以及多模型异常分数融合机制,在无监督或弱监督条件下实现对多维遥测异常的高灵敏度检测。

该方法的整体流程如图\ref{fig:Picture1}所示,主要包括多通道数据结构化、异步对齐与降采样、STL 时序分解、三类子模型独立训练以及异常分数融合与阈值判定五个阶段。

\begin{figure}[htbp]
\centering
\includegraphics[width=1\linewidth]{figures/c5/Picture1.png}
\caption{基于多维遥测时序STL分解的深度学习重构误差融合异常检测方法流程图}
\label{fig:Picture1}
\end{figure}

\begin{figure}[htbp]
\centering
\includegraphics[width=1\linewidth]{figures/c5/Picture2.png}
\caption{多维时序时间窗口降采样示意图}
\label{fig:Picture2}
\end{figure}

\begin{figure}[htbp]
\centering
\includegraphics[width=1\linewidth]{figures/c5/Picture3.png}
\caption{时序STL分解示意图}
\label{fig:Picture3}
\end{figure}

首先,在多通道遥测数据加载与结构化处理中,将星载产品的多个遥测通道统一整理为标准化时序数据表,并通过数据质量控制剔除误码帧、异常采样间隔及非业务数据。整理后的数据可表示为
\[
\mathbf{D}\in\mathbb{R}^{T\times(N_c+2)},
\]
其中每一行对应一个有效时间样本,包含时间戳、卫星标识以及 \(N_c\) 个遥测通道的观测值,为后续统一建模提供基础。

随后,为解决多通道遥测异步采样的问题,引入基于固定时间窗口的对齐与降采样方法。通过设定统一窗口长度 \(\Delta T\),将所有通道的观测映射到同一时间轴上,并在窗口内计算均值作为代表值,从而构建对齐后的多通道时序矩阵
\[
\mathbf{X}_{\mathrm{win}}\in\mathbb{R}^{K\times C}.
\]
该步骤在实现多通道同步的同时,天然具备降噪和平滑周期结构的作用,是后续分解与建模的基础。

在此基础上,对对齐后的多通道时序引入 STL(Seasonal-Trend decomposition using Loess)分解方法,将每个通道的时间序列拆分为趋势、周期和残差三部分:
\[
x_k^{(c)} = T_k^{(c)} + S_k^{(c)} + R_k^{(c)}。
\]
其中,趋势项刻画长期缓变特性,周期项反映系统固有的运行节律,残差项则包含高频扰动与局部异常。通过启用稳健 STL 分解模式,可有效抑制离群点对趋势与周期估计的影响,使分解结果更加稳定可靠。对所有通道分别执行 STL 分解后,可得到趋势矩阵 \(\mathbf{T}\)、周期矩阵 \(\mathbf{S}\) 与残差矩阵 \(\mathbf{R}\)。

针对三类具有不同统计特性和时间尺度的分量,方法分别设计了独立的异常建模策略。在趋势层面,采用基于 LSTM 的自编码器对长期时间依赖关系进行建模,通过学习正常趋势演化模式,在异常发生时产生显著的重构误差;在周期层面,利用一维卷积自编码器对固定周期内的局部波形结构进行建模,能够高效捕捉周期畸变或形态异常;而对于残差分量,由于其主要体现随机噪声特性,则采用统计分布建模方式,通过均值与方差估计获得标准化偏差作为异常量化指标。

在检测阶段,趋势模型与周期模型分别输出对应的重构误差序列 \(e_T(k)\) 与 \(e_S(k)\),残差模型则输出基于统计偏差的异常度量 \(e_R(k)\)。为在统一时间轴上形成综合判决,引入加权融合机制,将三类误差线性组合为单一异常分数:
\[
s(k)=\alpha_T e_T(k)+\alpha_S e_S(k)+\alpha_R e_R(k),
\]
其中 \(\alpha_T,\alpha_S,\alpha_R\) 为非负权重且满足和为 1。该融合分数能够同时反映趋势偏移、周期畸变以及噪声分布异常等多种异常形态。

最后,基于训练阶段正常数据的异常分数分布构建自适应阈值(如基于百分位数的方法),实现逐点异常判定。当
\[
s(k)>\tau
\]
时,判定该时间点为异常;否则视为正常。同时,通过对数据段边界附近样本进行排除,避免由于窗口不完整引入的误判。

综上,该方法通过“分解—建模—融合”的多层结构设计,在异常样本稀缺的条件下,能够从多维遥测时序中有效识别隐性、早期的星载产品性能异常,相较于传统阈值型方法,在检测灵敏度、稳定性与可解释性方面均具有显著优势。

\subsection{实验-基于模拟仿真遥测数据的异常检测实验与效果验证}

\begin{figure}[htbp]
\centering
\includegraphics[width=1\linewidth]{figures/c5/pic5.png}
\caption{正常数据(左列)与离群点异常数据(右列)的原始遥测时序、降采样时序、分解后趋势项时序、周期项时序、残差项时序、异常分数时序与检测结果时序示意图}
\label{fig:pic5}
\end{figure}

\begin{figure}[htbp]
\centering
\includegraphics[width=1\linewidth]{figures/c5/pic6.png}
\caption{整体漂移数据(左列)与斜率变化数据(右列)的原始遥测时序、降采样时序、分解后趋势项时序、周期项时序、残差项时序、异常分数时序与检测结果时序示意图}
\label{fig:pic6}
\end{figure}

\begin{figure}[htbp]
\centering
\includegraphics[width=1\linewidth]{figures/c5/pic7.png}
\caption{混合异常数据的原始遥测时序、降采样时序、分解后趋势项时序、周期项时序、残差项时序、异常分数时序与检测结果时序示意图}
\label{fig:pic7}
\end{figure}


\section{星载产品异常检测结果评估}
