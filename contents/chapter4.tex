% !TEX root = ../main.tex

\chapter{GNSS星载产品多维时序异常检测方法}

星载设备作为导航信号的物理源头,其运行状态直接决定了定位服务的精度与可靠性。设备的老化或故障将导致信号畸变,进而影响用户端定位。星载设备的健康状态主要由下行的多维遥测数据表征。鉴于故障模式具有渐进性、隐蔽性及强耦合性,传统的阈值监测方法难以满足早期预警需求。如何在故障显性化导致信号质量下降之前,利用遥测数据精准识别隐性异常,是提升系统完好性的关键。
基于此,本章将研究拓展至遥测数据域,聚焦于老化突发失效与未知硬件故障。针对遥测数据的高维非线性及周期性特征,分别提出了两种基于深度学习的检测方法,并利用基于仿真遥测数据进行了实验验证模型有效性。

\section{星载设备异常定义}

星载设备指集成在人造地球卫星或其他航天器上的各类功能性单元,包括但不限于科学探测仪器、通信载荷、导航设备、电源系统、姿态控制系统、推进系统、热控系统以及数据处理单元等\cite{singh2021astrosat,Ganesan2020Fault,Ustin2024Current}。这些设备协同工作,确保卫星能够成功执行其预定任务,例如对地观测、全球定位、空间科学实验或提供通信服务\cite{Kodheli2020Satellite}。星载设备的性能和可靠性直接影响到整个卫星系统的任务成功率和在轨寿命\cite{Tafazoli2009A}。随着卫星系统复杂性的增加,对星载设备的健康监测和也变得尤为重要。

星载设备的异常通常根据成因与表现形式被划分为五大类框架:设计与制造缺陷类、环境诱发类、老化与性能退化类、随机或未知硬件故障类,以及软件与操作失误类\cite{Sulaiman2025Radiation-induced,McKnight2017Examination}。其中,设计缺陷多源于工艺不足或元器件早期失效\cite{Loring2020Applications};环境诱发类主要涉及空间辐射、热循环及充放电效应导致的器件损伤\cite{Gubby2002Space,Nwankwo2020The}。老化类涉及设备在长期运行中的参数漂移与磨损\cite{Roesler2017Orbital,Nwankwo2020The};而随机或未知硬件故障则指无明显先验机理、仅在传感器遥测中表现为异常模式的故障\cite{Buitrago-Leiva2024Statistical}。

在上述分类体系下,本研究将监测对象聚焦于高精尖贵重星载设备中两类最具挑战性的故障形式:一是长期服役后的老化突发失效,二是机理复杂的未知硬件故障。首先,长期使用导致的老化突然失效是高可靠性部件面临的主要威胁。这类故障典型存在于原子钟、电源系统等核心组件中。文献指出,此类设备在经历长期的轴承磨损、电容老化等渐进式退化后,往往会在某一时刻出现功能的突发性丧失\cite{Aghaei2022Review,Chacko2025A}。这种“渐进退化导致突发失效”的模式,因其前兆隐蔽,常被归类为预警与健康管理的难点。其次,成因不明的未知硬件故障是另一难题,其失效机理(如微小结构损伤、容器泄露或接触不良)在地面难以完全复现或建模。在轨运行期间,此类故障只能通过分析加速度、功耗及温度等多源传感器的多变量时间序列数据来推断异常模式\cite{Yuan2021Fault,Barzegar2022Fault}。

高精尖星载设备的在轨故障往往伴随着极其高昂的经济代价与不可逆的战略损失\cite{Tetik2024A}。统计数据显示,仅电源系统一类设备的故障,在1990年至2006年间便导致商用与科学卫星累计损失约44亿美元\cite{Landis2006Causes}。这类故障通常表现为电池容量衰减或阵列机械卡死,直接导致整星能源供给不足,进而使任务提前终止,其经济后果等同于整星资产的直接报废\cite{Kim2012Spacecraft}。更宏观的预测模型显示,在极端空间环境如超级磁暴下,若缺乏有效的防护机制,大规模的关键载荷与平台设备失效可能引发高达700亿美元级别的行业性损失\cite{Odenwald2006Forecasting}。上述数据表明,随着卫星任务从试验性向业务关键型转变,单星或星座系统的关键设备失效已成为航天工程中巨大经济损失源头。

基于早期预警实施主动防护,是降低故障损失、实现设备全生命周期管理的核心手段\cite{Ibrahim2020Machine}。一旦监测系统捕捉到异常前兆,星上管理系统可立即采取负载降级、故障支路隔离或切换冗余单元等措施,甚至在极端情况下实施主动关机\cite{Ganesan2023Simultaneous,Wang2024A}。基于这种策略,能够将潜在的整星报废风险转化为部分能力受限的降级运行模式,不仅最大程度保留了剩余商业或科学价值\cite{Ibrahim2020Machine}。对于未来的在轨服务任务,保留完整的硬件实体为后续的机器人维修、模块更换或回收翻修提供了必要的物理基础\cite{Medina2017Towards}。

综上所述,鉴于高精尖星载设备故障带来的巨大经济风险,以及故障前兆的可检测性与主动防护的工程价值,传统的被动式事后分析已无法满足现代航天任务的需求。必须建立一套完善的星载设备健康管理与异常检测体系,特别是针对隐蔽性强、潜伏期长的隐性异常进行重点监测\cite{Tao2021Long-term}。

\section{现有星载设备异常检测方法与缺陷}
星载设备遥测数据的复杂特性决定了其异常检测的挑战性。首先,遥测数据具有极高的维度和复杂的时空相关性,单一设备的状态往往由电压、电流、温度等多源异构参数共同表征,且参数间存在非线性耦合关系。其次,卫星运行环境具有强动态性,轨道周期、光照条件及工作模式的切换会导致数据分布发生漂移与噪声,使得正常与异常的边界难以界定。此外,针对老化突发失效与未知硬件故障,其早期特征往往淹没在环境噪声中,表现为微弱的渐进式变化或未知的分布模式。因此,如何在资源受限的星载计算环境下,从海量、高噪、非线性的时序数据中精准识别异常,是当前研究的核心难点。

\begin{table*}[htbp]
\centering
\bicaption{星载设备异常检测方法的阶段划分、技术特点与局限性分析}{Phase classification, technical characteristics, and limitations of spacecraft onboard equipment anomaly detection methods.}
\renewcommand{\arraystretch}{1.2}
\begin{tabularx}{\textwidth}{p{2.2cm} p{2.6cm} X X}
\toprule
\textbf{阶段} &
\textbf{具体技术} &
\textbf{主要优势} &
\textbf{局限性与缺陷} \\
\midrule

\multirow{3}{=}{\textbf{基于统计学与模型方法}} 
& 阈值检测 
& 计算极简,实时性最高,可解释性强 
& 无法检测阈值内的上下文异常;难以适应环境漂移 \\

& 马氏距离 
& 考虑了变量间的相关性;计算代价低 
& 假设数据服从高斯分布;仅能处理线性相关性 \\

& 物理/机理模型 
& 理论依据充分,异常定位准确,无需大量训练数据 
& 模型构建成本复杂;难以覆盖未知的物理机制 \\

\midrule
\multirow{3}{=}{\textbf{经典机器学习方法}} 
& k-近邻 / LOF 
& 适用于多模态数据;无需预先假设数据分布 
& 高维空间中计算复杂度显著增长 \\

& 隔离森林 
& 处理速度快;适合大规模数据集;对异常敏感 
& 对时序相关性利用不足;对局部微小漂移不敏感 \\

& 支持向量机 
& 适合小样本或不平衡数据;定义明确的正常边界 
& 对噪声敏感;超参数选择困难;难以捕捉长期依赖 \\

\midrule
\multirow{3}{=}{\textbf{深度学习方法}} 
& LSTM / RNN 
& 能够捕捉长期时间依赖;适合非线性时序数据 
& 串行计算导致训练慢;在极长序列存在梯度消失问题 \\

& AE / VAE 
& 无监督学习(无需标签);自动提取非线性特征 
& 容易过拟合;异常解释性较差 \\

& GAN 生成模型 
& 能生成高质量的模拟数据;对复杂分布建模能力强 
& 训练不稳定;难以收敛;计算资源消耗大 \\

\bottomrule
\end{tabularx}
\label{tab:anomaly_detection_methods_comparison}
\end{table*}


早期的异常检测技术主要依赖于统计学理论与物理模型,侧重于检测数据对预设分布或规则的偏离。基于阈值的方法 (Threshold-based Techniques) 是工程应用中最直观的手段,通过设定参数的上下限(如3-sigma准则)来识别越界异常,已在北斗导航卫星系统的星历数据一致性检查中得到应用 \cite{cai2024evaluation, murphy2023overview}。为捕捉变量间的相关性,马氏距离 (Mahalanobis Distance) 被广泛引入,它通过计算数据点到分布中心的协方差加权距离来度量异常程度,因计算代价低而适用于星载实时诊断 \cite{katsube2025towards}。此外,自回归积分滑动平均模型 (ARIMA) 等时间序列模型通过预测未来值并将预测残差作为判据,而基于物理机理的方法则试图建立设备的理想“健康基线” \cite{li2019research}。尽管这些方法具有可解释性强、计算量小的优势,但其极度依赖专家先验知识,且难以适应多变量耦合的非线性系统,在面对未知故障模式时往往力不从心。

为了减少对人工规则的依赖,经典机器学习算法被引入以挖掘数据中的潜在模式,主要分为基于距离/密度的方法和基于边界的方法。基于距离的方法如 k-近邻 (k-NN),通过计算样本与其最近邻居的欧氏距离来识别离群点 \cite{bieber2023generic}。基于密度的算法如局部离群因子 (LOF) 和核密度估计 (KDE),则侧重于识别处于低密度区域的样本,对于密度不均匀的数据集具有较好的适应性 \cite{yu2022edge, guo2023contrastive}。在无监督学习领域,主成分分析 (PCA) 利用降维重构误差来检测异常,而隔离森林 (Isolation Forest) 则利用异常点“稀少且不同”的特性,通过构建随机树将异常点快速剥离 \cite{bieber2023generic}。此外,单类支持向量机 (OC-SVM) 通过学习包围正常数据的最小超球面边界,有效地实现了半监督下的异常判别 \cite{yu2024amad}。然而,此类方法在特征工程方面仍需人工干预,且在处理具有长时序依赖的复杂遥测数据时,往往因无法捕捉时间特征而出现性能瓶颈。

随着深度神经网络的发展,基于重构和预测的深度学习模型已成为当前解决高维时序异常检测的主流手段。预测类方法主要利用长短期记忆网络 (LSTM) 等循环神经网络 (RNN) 捕捉数据的时间依赖性,通过比较下一时刻的预测值与观测值的误差来判定异常,例如 IF-TEA-LSTM 模型通过结合隔离森林与 LSTM 显著提升了导航卫星数据的检测精度 \cite{cai2024evaluation, murphy2023overview}。重构类方法则以自编码器 (Autoencoder, AE) 及其变体为核心,假设模型无法有效压缩和还原未见过的异常模式,从而利用重构误差作为异常分数 \cite{murphy2023overview}。其中,变分自编码器 (VAE) 引入了概率分布建模,如 OmniAnomaly 模型结合峰值超越阈值 (POT) 方法,实现了更鲁棒的无监督检测 \cite{yu2024amad}。此外,生成对抗网络 (GAN) 通过生成器与判别器的博弈学习正常分布,也被证实能有效区分异常样本。这些方法具备强大的特征自学习能力,能够处理复杂的非线性映射关系。

尽管现有方法取得了显著进展,但在应对高精尖设备“老化突发失效”和“未知硬件故障”时仍存在明显局限,难以满足高可靠性任务的需求。首先,传统统计与机器学习方法在高维时空数据面前表现乏力,难以捕捉微弱的早期退化特征;其次,基础的深度学习模型(如 LSTM、AE)虽然提升了非线性拟合能力,但在面对极其稀有的未知故障样本时,容易出现过拟合或泛化能力不足的问题。更重要的是,现有大多数方法缺乏对“隐性异常”的针对性设计,难以在故障显性化之前实现有效的早期预警。因此,本研究旨在探索一种能够融合深层时空特征、具备强抗噪能力且对未知模式敏感的新型异常检测架构,以解决关键星载设备早期故障预警难的问题。

\section{基于数据多维遥测时序信号的异常检测方法}

\subsection{基于融合多窗口统计特征的LSTM自编码器遥测数据异常检测}

在前文的综述中指出,尽管深度学习方法在处理高维遥测数据时展现出优于传统统计方法的潜力,但在实际工程应用中仍面临原始遥测数据通常伴随高频噪声的问题,直接将原始信号输入网络容易导致模型过拟合噪声,而非学习到真正的设备运行模式。同时,单纯的空间特征提取模型忽略了数据的时间依赖性,而传统的有监督时序模型则受限于异常标签的极度匮乏,难以应对未知故障。针对上述问题,本节提出一种基于融合多窗口统计特征的LSTM自编码器异常检测方法。该方法引入多窗口统计特征提取作为前置增强模块,利用不同尺度的滑动窗口提取均值、方差等统计量,以实现不同长程特征的获取。随后,利用LSTM构建自编码器,在潜在空间中对多维统计特征的时序演化进行建模。

\subsubsection{算法流程设计}

针对星载产品多通道遥测数据维度高、时间跨度长且异常样本稀缺的问题,本文提出一种基于时序自编码器的多维时序异常检测方法。该方法在保证时间连续性的前提下,对遥测数据进行分段建模与多尺度特征提取,并通过无监督时序自编码器学习正常行为模式,最终基于重构误差实现异常检测。方法整体流程如图\ref{fig:pat1_pic1}所示,主要包括数据预处理、时间分段、特征构建、时序建模以及异常判定五个步骤。

\begin{figure}[htbp]
\centering
\includegraphics[width=0.7\linewidth]{figures/c4/patent1/pic1.png}
\bicaption{基于融合多窗口统计特征的 LSTM 自编码器遥测数据异常检测方法流程图}{Flowchart of the LSTM autoencoder–based telemetry anomaly detection method using fused multi-window statistical features。}
\label{fig:pat1_pic1}
\end{figure}


首先,对星载产品下行的多通道遥测数据进行统一获取与质量预处理,剔除误码帧、异常采样帧以及非业务采样数据,仅保留采样时间间隔规律的有效遥测序列。设质量过滤后的多通道遥测数据表示为
\begin{equation}
    \mathbf{X}_{\mathrm{valid}}
=
\left\{
\left(
x_t^{(1)}, x_t^{(2)}, \ldots, x_t^{(N_c)}
\right)
\right\}_{t=1}^{T}
\end{equation}
其中,$x_t^{(c)}$ 表示第 $c$ 个遥测通道在时刻 $t$ 的有效观测值。

为避免后续处理跨越非连续时间区间,依据相邻时间戳间隔对遥测序列进行自动分段。通过计算相邻时间差 $\Delta t_i = t_{i+1}-t_i$,当其超过阈值 $T_{\mathrm{gap}}$ 时识别为分段边界,从而将完整遥测序列划分为若干连续数据段 $S_j$。后续所有滑动窗口、统计特征计算以及序列构建操作均严格限制在单一数据段内完成,以避免时间不连续性对建模结果的干扰。

在时间分段约束下,对各遥测通道进行多尺度平滑去噪与趋势提取。如图\ref{fig:pat1_pic2}所示,设定一组平滑窗口尺度,对原始信号计算滑动平均以抑制高频噪声,并通过信噪比增益指标选择各通道的主平滑尺度。在此基础上,引入多窗口滚动统计方法,从平滑信号中提取滚动均值、标准差、极值、变化范围以及一阶变化率等统计特征。对于任一时刻 $k$,将所有通道的统计特征进行拼接,构成整体特征向量 $\mathbf{f}_k$,从而得到特征矩阵

\begin{equation}
    \mathbf{X}_{\mathrm{features}}
=
\left[
\mathbf{f}_1, \mathbf{f}_2, \ldots, \mathbf{f}_K
\right]^{\mathrm{T}}
\in \mathbb{R}^{K \times D_{\mathrm{total}}}
\end{equation}

随后基于训练数据计算特征均值与标准差,对特征进行标准化处理,以保证不同特征维度在同一量纲下参与建模。

\begin{figure}[htbp]
\centering
\includegraphics[width=1\linewidth]{figures/c4/patent1/pic2.png}
\bicaption{基于融合多窗口统计特征的 LSTM 自编码器检测实验数据预处理与特征矩阵构建方法}{Data preprocessing and feature matrix construction for detection experiments using an LSTM autoencoder with fused multi-window statistical features.}
\label{fig:pat1_pic2}
\end{figure}

如图\ref{fig:pat1_pic3}所示,在完成特征构建与标准化后,引入基于循环神经网络的时序自编码器对多维特征序列进行无监督建模。在每个数据段内,采用滑动窗口方式构造长度为 $L_{\mathrm{seq}}$ 的输入序列:
\begin{equation}
    \mathbf{X}_{\mathrm{seq}}
=
\left[
\mathbf{f}_i^{\mathrm{norm}},
\mathbf{f}_{i+1}^{\mathrm{norm}},
\ldots,
\mathbf{f}_{i+L_{\mathrm{seq}}-1}^{\mathrm{norm}}
\right]
\end{equation}
并约束序列不跨越段边界。序列样本输入时序自编码器,经编码器映射至低维潜在空间,再由解码器重构得到输出序列,其映射关系表示为
\begin{equation}
    \mathbf{X}_{\mathrm{seq}}
\;\xrightarrow{\;\mathcal{E}_{\theta}\;}\;
\mathbf{z}
\;\xrightarrow{\;\mathcal{D}_{\phi}\;}\;
\hat{\mathbf{X}}_{\mathrm{seq}}
\end{equation}
模型通过最小化输入序列与重构序列之间的均方误差进行训练,从而学习系统在正常工况下的时序演化特征。

\begin{figure}[htbp]
\centering
\includegraphics[width=1\linewidth]{figures/c4/patent1/pic3.png}
\bicaption{时序自编码器模型设计与LSTM单元结构}{Design of the temporal autoencoder model and LSTM unit structure.}
\label{fig:pat1_pic3}
\end{figure}


在测试阶段,针对每个时间点构造以该点为终点的输入序列,并计算对应的重构误差。由重构结果中提取时间点 $k$ 的重构特征向量 $\hat{\mathbf{f}}_k^{\mathrm{norm}}$,定义逐点重构误差为
\begin{equation}
    e_k
=
\left\|
\mathbf{f}_k^{\mathrm{norm}}
-
\hat{\mathbf{f}}_k^{\mathrm{norm}}
\right\|_2^2
\end{equation}
为提高异常评分的鲁棒性,对重构误差序列进行滑动平均平滑,得到异常评分序列 $\{s_k\}$。

最后,基于训练数据对应的异常评分分布构建正常参考分布,并采用百分位数法或固定阈值法确定异常判定阈值 $\tau$。当异常评分超过阈值时判定为异常;同时在数据段起始与结束位置引入边界排除机制,以减少由窗口不完整引起的伪异常,从而获得最终稳定可靠的异常检测结果。

\subsubsection{数据集与实验设计}

为了在可控且完备的条件下验证所提方法的有效性,本研究构建了一套基于中地球轨道卫星真实遥测特征的高保真仿真数据集。该数据集模拟了星载单机设备在为期一年内的运行状态,采样频率设定为每分钟一次。数据集共包含 5 路遥测通道,分别表示不同的设备健康状态表征物理量。数据总量为 $5.25 \times 10^5$ 个采样点。在时间维度上,数据并非连续完整,而是包含了模拟下行链路中断或卫星过境间隙造成的随机缺失,形成了 150 余个长短不一的有效数据段,旨在复现真实在轨环境下的非连续观测特征。

为了系统性地评估模型对不同故障类型的检测能力,在仿真数据中注入了四种具有明确物理意义的典型异常模式。 
\begin{itemize}
\item 斜率变化异常:模拟星载设备性能发生渐进性退化的早期信号,表现为时序数据变化速率的微小改变。 
\item 离群点异常:模拟数据剧烈跳变,表现为单个或短时的数值突变。 
\item 整体漂移异常:模拟系统性偏差,表现为数据均值或基准线的整体平移。 
\item 组合异常:由上述三种异常在时域上的叠加或连续发生构成。 
\end{itemize}

如图\ref{fig:pat1_pic4}所示,数据集的训练集部分选取仿真时间轴的前 80\% 数据,包含约 $4.2 \times 10^5$ 个样本点,该部分数据不含任何注入的异常。实验希望模型仅学习设备在健康状态下的时序演化规律与多变量耦合关系,构建出正常行为基线。 测试集部分选取仿真时间轴的后 20\% 数据,包含约 $1.05 \times 10^5$ 个样本点。该数据集为混合场景,既包含了延续正常趋势的数据,也包含了人工注入的各类异常数据。测试集用于模拟模型未曾见过的未来运行状态,重点考察模型在面对未知数据时的重构能力与判别精度。

\begin{figure}[htbp]
\centering
\includegraphics[width=1\linewidth]{figures/c4/patent1/pic4.png}
\bicaption{基于融合多窗口统计特征的 LSTM 自编码器实验的训练数据与测试数据。其中训练数据为模拟星载产品遥测时序数据(左上),测试数据包含正常数据(左中)、斜率漂移异常数据(左下)、离群点异常数据(右上)、整体漂移异常数据(右中)、混合异常数据(右下)}{Training and test data used in the LSTM autoencoder experiment based on fused multi-window statistical features. The training data consist of simulated spacecraft onboard product telemetry time-series data (upper left), while the test data include normal data (middle left), slope drift anomaly data (lower left), outlier anomaly data (upper right), overall drift anomaly data (middle right), and mixed anomaly data (lower right)..}
\label{fig:pat1_pic4}
\end{figure}

本章实验沿用第三章中二分类检测的三个核心评价指标:精确率(Precision)、召回率(Recall)和 F1 分数(F1-Score)。在星载设备异常检测任务中,通常关注的是对故障状态的捕捉能力,因此将异常样本定义为正例,将正常样本定义为负例。

\subsubsection{实验结果与分析}

通过将测试集数据输入训练好的模型,计算逐点重构误差并经过平滑处理得到异常评分序列。图 \ref{fig:pat1_pic5} 至图 \ref{fig:pat1_pic7} 展示了模型在正常、渐进退化、突发干扰及复合故障等四种典型场景下的响应曲线。

\begin{figure}[htbp]
\centering
\includegraphics[width=1\linewidth]{figures/c4/patent1/pic5.png}
\bicaption{基于融合多窗口统计特征的 LSTM 自编码器实验的正常数据(左)与斜率漂移异常数据(右)的原始遥测时序、平滑后时序与异常得分时序}{Raw telemetry time series, smoothed time series, and anomaly score time series for normal data (left) and slope drift anomaly data (right) in the LSTM autoencoder experiment based on fused multi-window statistical features.}
\label{fig:pat1_pic5}
\end{figure}

模型在正常状态下的稳定性是衡量其可用性的首要标准。如图 \ref{fig:pat1_pic5}(左侧)所示,当输入的模拟遥测数据处于符合预期的正常缓变趋势时(表现为按既定物理规律的缓慢下降或周期波动),模型输出的异常分数表现出极高的稳定性。具体而言,在无异常注入的时间段内,异常评分的均值维持在 0.5 左右,最大峰值严格控制在 1.5 以下,远低于预设的判定阈值($\tau=4.0$)。这一结果表明,通过无监督学习,LSTM自编码器已成功捕获了多维遥测变量间的正常耦合关系与演化规律,能够有效抑制因环境噪声或正常工况波动引起的误报,确立了稳健的“健康基线”。

针对隐蔽性最强的设备老化征兆——斜率变化异常,本方法展现了有效的早期识别能力。如图 \ref{fig:pat1_pic5}(右侧)所示,在模拟退化开始的初始阶段,异常分数并未立即突变,而是随着观测窗口内“趋势偏离度”的累积,呈现出单调递增的态势。实验数据显示,在异常注入点之后的约 100 个时间步内,异常分数显著突破阈值并持续上升。虽然存在有限的检测延迟,但这种延迟是模型确认趋势改变统计显著性的必要过程。更重要的是,异常分数的增长幅度与故障持续时间呈正相关,这一特性与设备性能随时间累积退化的物理机理高度吻合,证明了该指标能够定量反映老化故障的严重程度,为地面实施早期预警提供了充足的时间窗口。

针对离群点与整体漂移两类显著异常,模型表现出不同的响应机制。对于瞬时大幅度离群点,如图\ref{fig:pat1_pic6} 左侧所示,模型实现了即时响应,异常分数在故障时刻形成尖锐的脉冲峰值,精准定位了干扰发生的时间点。值得注意的是,在长持续时间的离群段中部,异常分数出现了回落现象。这揭示了LSTM网络特有的记忆适应机制,随着异常状态的持续,模型逐渐将这种新的稳态理解为一种发生了局部漂移的准正常模式。这种机制证明模型具备深层的时序语义理解能力,而非简单的阈值比较。

\begin{figure}[htbp]
\centering
\includegraphics[width=1\linewidth]{figures/c4/patent1/pic6.png}
\bicaption{基于融合多窗口统计特征的 LSTM 自编码器实验的离群点异常数据(左)与整体漂移异常数据(右)的原始遥测时序、平滑后时序与异常得分时序}{Raw telemetry time series, smoothed time series, and anomaly score time series for outlier anomaly data (left) and overall drift anomaly data (right) in the LSTM autoencoder experiment based on fused multi-window statistical features.}
\label{fig:pat1_pic6}
\end{figure}

对于系统性偏差,模型表现出极高的灵敏度。如图\ref{fig:pat1_pic6} 右侧所示,异常分数在漂移发生的跳变沿处达到峰值(Score $\approx$ 160),并在随后的持续漂移阶段稳定维持在 10-20 的高位区间。这种显著的台阶式响应清晰地将异常状态与正常背景分离开来,验证了方法在应对因元器件老化或环境突变导致的系统性偏差时具有极强的鲁棒性。

在多故障耦合的混合场景下,本方法体现了优异的综合检测性能。如图 \ref{fig:pat1_pic7} 所示,测试序列包含 5 处离散离群点与 1 段叠加的斜率变化。实验结果表明,除斜率变化起始阶段极其微弱的信号外,方法对其余所有异常均实现了 100\% 的检出。特别是在离群点与斜率变化发生重叠的时间区域,异常分数呈现出明显的协同增强效应,即复合故障产生的重构误差显著高于单一故障之和。这一结果证明,本方法生成的异常分数不仅可用于定性判别,还能有效表征复合故障场景下的综合健康风险。

\begin{figure}[htbp]
\centering
\includegraphics[width=0.5\linewidth]{figures/c4/patent1/pic7.png}
\bicaption{基于融合多窗口统计特征的 LSTM 自编码器实验的混合异常数据的原始遥测时序、平滑后时序与异常得分时序}{Raw telemetry time series, smoothed time series, and anomaly score time series for mixed anomaly data in the LSTM autoencoder experiment based on fused multi-window statistical features.}
\label{fig:pat1_pic7}
\end{figure}

为排除单一样本的偶然性并深入剖析模型的分类细节,本节基于测试集(共计 105,000 个样本点)构建了二分类混淆矩阵。根据前述实验设定,测试集中包含真实异常样本 3,200 个,正常样本 101,800 个。表 \ref{tab:confusion_matrix} 展示了模型在测试集上的具体分类结果统计。

基于表 \ref{tab:confusion_matrix} 中的基础统计量,进一步计算得到本方法的各项核心性能指标,如表 \ref{tab:performance_metrics_formula} 所示。

\begin{table}[htbp]
\centering
\bicaption{基于融合多窗口统计特征的 LSTM 自编码器异常检测模型的混淆矩阵统计结果}{Confusion matrix statistics of the LSTM autoencoder–based anomaly detection model using fused multi-window statistical features.}
\renewcommand{\arraystretch}{1.5}
\begin{tabularx}{0.9\textwidth}{p{3.2cm} X X X}
\toprule
\textbf{真实标签} &
\textbf{预测结果:正常} &
\textbf{预测结果:异常} &
\textbf{合计} \\
\midrule

正常 
& TN = 101{,}794 (真负例) 
& FP = 6 (误报) 
& 101{,}800 \\

异常 
& FN = 91 (漏报) 
& TP = 3{,}109 (命中) 
& 3{,}200 \\

\midrule
合计 
& 101{,}885 
& 3{,}115 
& 105{,}000 \\

\bottomrule
\end{tabularx}
\label{tab:confusion_matrix}
\end{table}

\begin{table}[htbp]
\centering
\bicaption{基于融合多窗口统计特征的 LSTM 自编码器异常检测模型的主要性能指标及计算结果}{Main performance metrics and calculation results of the LSTM autoencoder–based anomaly detection model using fused multi-window statistical features.}
\renewcommand{\arraystretch}{1.2}
\begin{tabularx}{0.9\textwidth}{p{4.2cm} X p{3cm}}
\toprule
\textbf{性能指标} &
\textbf{计算公式} &
\textbf{实验数值} \\
\midrule

精确率(Precision) 
& $\frac{TP}{TP + FP}$ 
& 99.81\% \\

召回率(Recall) 
& $\frac{TP}{TP + FN}$ 
& 97.16\% \\

F1 分数(F1-Score) 
& $2 \cdot \frac{P \cdot R}{P + R}$ 
& 98.47\% \\

\bottomrule
\end{tabularx}
\label{tab:performance_metrics_formula}
\end{table}


实验数据显示,在 10 万余个正常样本中,仅产生了 6 次误报。这表明模型对环境噪声具有极强的过滤能力,极大地降低了地面运控人员处理无效警报的负担。同时,模型具有极高的查全率,成功捕获了 3200 个异常中的 3109 个。漏报样本(FN=91)主要集中在渐进性斜率变化异常发生的最初始阶段,这是由于趋势特征尚未显著累积所致,属于算法原理上的合理延迟。

\subsection{基于STL分解的LSTM-CNN混合自编码器异常检测}
星载设备的在轨遥测数据并非无序的随机波动,其时序特征受到卫星轨道动力学与空间环境的共同制约。由于卫星绕地球运行的周期性规律,绝大多数关键遥测参数在时域上呈现出显著的周期性。在实际工程中,这种周期信号同时叠加了由设备老化引起的长期趋势项以及传感器高频随机噪声。面对这种“周期性+趋势性+噪声干扰”强耦合的多维非平稳时间序列,前文所述的单一全连接或循环神经网络模型往往无法较好捕捉特征。单一模型在训练过程中容易过度拟合高频周期波动,而忽略了掩盖在大幅度周期变化之下的微弱趋势性偏移,同时难以从强背景噪声中剥离出真实的波形畸变。

为了解决上述问题,本节提出一种解耦并分治的基于STL分解的LSTM-CNN混合自编码器结构。其核心思想是利用STL时序分解算法将复杂的混合遥测信号分离为趋势分量、季节分量和残差分量。在此基础上,分别利用LSTM网络来建模趋势项,利用CNN网络来建模周期项。

\subsubsection{STL时序分解算法}
STL (Seasonal-Trend decomposition using Loess) 是一种由Cleveland等人提出的、基于局部加权回归 (Locally Weighted Scatterplot Smoothing, LOESS) 的时间序列分解方法\cite{cleveland1990stl}。与传统的基于移动平均或参数模型的分解技术不同,STL 方法具有高度的灵活性和鲁棒性,能够有效应对非线性趋势和复杂的季节性变化。

理论上,任一时间序列 $Y_t$ 都可以通过STL加性分解为三个分量:趋势分量 (Trend)、季节分量 (Seasonal) 和残差分量 (Residual)\cite{peixeiro2022time}。其数学表达如下:

\begin{equation}
    Y_t = T_t + S_t + R_t, \quad t = 1, \dots, N
\end{equation}
其中:
\begin{itemize}
    \item $Y_t$ 表示在时刻 $t$ 的观测值;
    \item $T_t$ 代表趋势分量;
    \item $S_t$ 代表季节分量;
    \item $R_t$ 为去除趋势和季节后的残差项。
\end{itemize}

STL 的核心思想是通过嵌套的内循环和外循环迭代计算各分量。内循环主要负责趋势项和季节项的拟合。在每一次迭代中,算法首先对季节子序列进行 LOESS 平滑以估计季节分量 $S_t$;随后,通过对去季节后的序列 $Y_t - S_t$ 应用 LOESS 平滑来估计趋势分量 $T_t$。外循环主要用于增强算法对异常值的鲁棒性。算法根据内循环产生的残差项 $R_t$ 计算鲁棒性权重 $\rho_t$。对于残差较大的观测点,赋予较小的权重,从而在下一轮内循环的平滑过程中降低其对趋势和季节估计的影响。这种双循环机制使得 STL 不仅能够捕捉随时间动态变化的季节性模式,还能有效抵抗极端值的干扰。

在本研究中采用 STL 方法主要基于以下优势:
\begin{itemize}
    \item 动态季节性处理能力:STL 允许季节分量随时间发生缓慢演变,而非强制假设季节性是固定不变的周期函数,这对于长跨度的时间序列分析尤为重要。
    \item 鲁棒性:通过外循环的抗差权重机制,STL 能够有效防止数据中的瞬时突变或测量误差扭曲趋势项的估计。
    \item 参数灵活性:研究者可以根据数据的具体特性,通过调整季节窗口和趋势窗口的平滑参数,精准控制各分量的平滑程度。
\end{itemize}

\subsubsection{算法流程设计}
针对星载产品遥测数据维度高、时间跨度长、异常样本极度稀缺等特点,传统依赖固定阈值或单变量统计特征的方法往往只能识别幅度显著的显性异常,难以及时发现隐性的、早期的性能退化问题。为此,提出了一种基于深度学习的星载产品分段多维遥测时序异常检测方法。该方法通过引入分段管理、多尺度去噪、趋势–周期–残差解耦建模以及多模型异常分数融合机制,在无监督或弱监督条件下实现对多维遥测异常的高灵敏度检测。

该方法的整体流程如图\ref{fig:Picture1}所示,主要包括多通道数据结构化、异步对齐与降采样、STL 时序分解、三类子模型独立训练以及异常分数融合与阈值判定五个阶段。

\begin{figure}[htbp]
\centering
\includegraphics[width=1\linewidth]{figures/c4/Picture1.png}
\bicaption{基于STL分解的混合神经网络异常检测方法流程图}{Flowchart of the hybrid neural network–based anomaly detection method using STL decomposition.}
\label{fig:Picture1}
\end{figure}

首先,在多通道遥测数据加载与结构化处理中,将星载产品的多个遥测通道统一整理为标准化时序数据表,并通过数据质量控制剔除误码帧、异常采样间隔及非业务数据。整理后的数据可表示为
\begin{equation}
    \mathbf{D}\in\mathbb{R}^{T\times(N_c+2)}
\end{equation}
其中每一行对应一个有效时间样本,包含时间戳、卫星标识以及 \(N_c\) 个遥测通道的观测值,为后续统一建模提供基础。

随后,为解决多通道遥测异步采样的问题,引入基于固定时间窗口的对齐与降采样方法,如图\ref{fig:Picture2}所示。通过设定统一窗口长度 \(\Delta T\),将所有通道的观测映射到同一时间轴上,并在窗口内计算均值作为代表值,从而构建对齐后的多通道时序矩阵
\begin{equation}
    \mathbf{X}_{\mathrm{win}}\in\mathbb{R}^{K\times C}
\end{equation}
该步骤在实现多通道同步的同时,天然具备降噪和平滑周期结构的作用,是后续分解与建模的基础。

\begin{figure}[htbp]
\centering
\includegraphics[width=1\linewidth]{figures/c4/Picture2.png}
\bicaption{多维时序时间窗口降采样示意图}{Schematic illustration of time-window downsampling for multivariate time series.}
\label{fig:Picture2}
\end{figure}

在此基础上,对对齐后的多通道时序引入 STL(Seasonal-Trend decomposition using Loess)分解方法,将每个通道的时间序列拆分为趋势、周期和残差三部分:
\begin{equation}
    x_k^{(c)} = T_k^{(c)} + S_k^{(c)} + R_k^{(c)}
\end{equation}
其中,趋势项刻画长期缓变特性,周期项反映系统固有的运行节律,残差项则包含高频扰动与局部异常。通过启用稳健 STL 分解模式,可有效抑制离群点对趋势与周期估计的影响,使分解结果更加稳定可靠。对所有通道分别执行 STL 分解后,可得到趋势矩阵 \(\mathbf{T}\)、周期矩阵 \(\mathbf{S}\) 与残差矩阵 \(\mathbf{R}\)。

针对三类具有不同统计特性和时间尺度的分量,方法分别设计了独立的异常建模策略。在趋势层面,采用基于 LSTM 的自编码器对长期时间依赖关系进行建模,通过学习正常趋势演化模式,在异常发生时产生显著的重构误差;在周期层面,利用一维卷积自编码器对固定周期内的局部波形结构进行建模,能够高效捕捉周期畸变或形态异常;而对于残差分量,由于其主要体现随机噪声特性,则采用统计分布建模方式,通过均值与方差估计获得标准化偏差作为异常量化指标。

在检测阶段,趋势模型与周期模型分别输出对应的重构误差序列 \(e_T(k)\) 与 \(e_S(k)\),残差模型则输出基于统计偏差的异常度量 \(e_R(k)\)。为在统一时间轴上形成综合判决,引入加权融合机制,将三类误差线性组合为单一异常分数:
\begin{equation}
    s(k)=\alpha_T e_T(k)+\alpha_S e_S(k)+\alpha_R e_R(k)
\end{equation}
其中 $(\alpha_T,\alpha_S,\alpha_R)$ 为非负权重且满足和为 1。该融合分数能够同时反映趋势偏移、周期畸变以及噪声分布异常等多种异常形态。

最后,基于训练阶段正常数据的异常分数分布构建自适应阈值(如基于百分位数的方法),实现逐点异常判定。当$s(k)>\tau$ 时,判定该时间点为异常;否则视为正常。同时,通过对数据段边界附近样本进行排除,避免由于窗口不完整引入的误判。

综上,该方法通过“分解—建模—融合”的多层结构设计,在异常样本稀缺的条件下,能够从多维遥测时序中有效识别隐性、早期的星载产品性能异常,相较于传统阈值型方法,在检测灵敏度、稳定性与可解释性方面均具有显著优势。

\subsubsection{数据集与实验设计}
与前述实验中使用的通用遥测时序数据不同,本实施例构建的数据集旨在专门模拟中地球轨道卫星在复杂空间环境下的准周期性运行特征。虽然同样包含斜率变化、离群点、整体漂移及组合异常四种典型故障模式,但本数据集在基础信号仿真上不再是单一的随机过程,而是由主周期为 13 小时的复合正弦波叠加而成,并混入了线性和对数形式的长期趋势项以及信噪比约为 30\% 的高斯白噪声。

在数据预处理层面,本实验采用了与前一节完全不同的异步对齐与时频分解策略。前一节实验侧重于多窗口统计特征的提取,而本实验首先引入基于绝对中位差算法的初级离群点剔除,随后以 30 分钟为固定窗口对 16 秒采样率的原始数据进行降采样对齐。这一步骤不仅实现了多通道数据的时序同步,更重要的是为 STL 分解提供了稳定的时间基准。基于此,STL 分解的关键参数季节周期窗口长度被精确设定为 $P=26$,从而确保算法能够从物理层面精准剥离出轨道周期分量,而非单纯依赖数据驱动的特征拟合。

本实验根据 STL 分解后的分量特性(如图\ref{fig:Picture3}所示)构建了三个独立的子模型: 
\begin{itemize} 
\item 趋势模型:沿用 LSTM 自编码器架构,但仅专注于低频趋势分量,旨在捕捉长达一年的缓慢老化特征。 
\item 周期模型:引入 1D-CNN自编码器 ,利用卷积核的局部感知能力提取 13 小时轨道周期内的波形形态特征,这是 LSTM 难以高效处理的。 
\item 残差模型:摒弃深度神经网络,转而采用基于高斯分布的统计模型,专门处理去相关后的随机噪声分量。 \end{itemize} 
所有深度学习子模型的输入序列长度均设为 26,隐藏层维度统一为 64,使用 Adam 优化器迭代训练 30 轮。

\begin{figure}[htbp]
\centering
\includegraphics[width=1\linewidth]{figures/c4/Picture3.png}
\bicaption{时序STL分解示意图}{Schematic illustration of STL decomposition for time series.}
\label{fig:Picture3}
\end{figure}

与前一节实验直接使用重构误差作为单一判据不同,本实验设计了加权融合判决机制。考虑到不同分量对不同类型异常的敏感度差异,实验设定了差异化的融合权重:趋势项权重 $\alpha_T=0.3$ 用于响应老化,周期项权重 $\alpha_S=0.2$ 用于监测波形畸变,残差项权重 $\alpha_R=0.5$ 用于捕捉突发扰动。最终的异常判定阈值基于训练集融合分数的分布特征固定为 $\tau=6.0$。这一策略旨在验证多视角融合机制是否能在复杂场景下比单一视角的重构误差提供更鲁棒的检测结果。

\subsubsection{实验结果与分析}
在未注入异常的正常测试集中,该方法展现了优异的信号解耦能力与基线稳定性,证明了其在强周期背景下的极低虚警率。如图 \ref{fig:pic5} 所示,尽管原始遥测信号包含大幅度的13小时轨道周期波动及显著的背景噪声,模型输出的融合异常分数始终维持在低位,即均值远小于 1.0,且全程低于预设阈值($\tau=6.0$)。这一结果表明,STL分解有效地将正常的周期波动与长期趋势剥离,使得 1D-CNN 和 LSTM 子模型能够分别精准重构周期波形与趋势基线。模型成功避免了将正常的轨道周期波动误判为异常,建立了一个稳健的健康基线。

\begin{figure}[htbp]
\centering
\includegraphics[width=1\linewidth]{figures/c4/pic5.png}
\bicaption{基于STL分解的混合神经网络异常检测实验的正常数据(左列)与离群点异常数据(右列)的原始遥测时序、降采样时序、分解后趋势项时序、周期项时序、残差项时序、异常分数时序与检测结果时序示意图}{Schematic illustration of the raw telemetry time series, downsampled time series, decomposed trend component, seasonal component, residual component, anomaly score time series, and detection result time series for normal data (left column) and outlier anomaly data (right column) in the STL decomposition–based hybrid neural network anomaly detection experiment.}
\label{fig:pic5}
\end{figure}

针对模拟的离群点异常,如图 \ref{fig:pic5} 所示,方法实现了对短时高频突变的即时发现。这主要得益于预处理阶段引入的基于 MAD 的鲁棒统计标记,结合残差分量模型对高斯分布偏离的敏感性,使得异常发生瞬间融合分数即产生尖峰报警。

当数据发生阶跃式漂移时,如图 \ref{fig:pic6} 所示,方法基于信号的实现了高置信度报警。漂移导致STL分解后的趋势项连续性与周期项波形结构同时受损,致使趋势与周期模型的重构误差同步激增。融合异常分数在漂移点迅速跳变至 10.0 以上,远超阈值,并在持续漂移阶段保持高位,验证了方法对系统性偏差的强捕捉能力。

针对最难检测的渐进式斜率变化,本方法通过趋势解耦机制成功克服了强周期信号的掩盖效应。如图 \ref{fig:pic6} 所示,实验表明在异常注入初期,虽然偏差幅度远小于轨道周期波动幅度,但异常分数仍呈现出缓慢上升的态势;随着退化程度加深,异常分数稳定突破阈值并呈单调递增。这种特性证实,通过STL剥离高频周期与噪声后,LSTM趋势模型能够专注于低频段的演化规律,从而敏锐捕捉到长期物理性能的微小偏离,实现了对星载产品老化的早期预警。

\begin{figure}[htbp]
\centering
\includegraphics[width=1\linewidth]{figures/c4/pic6.png}
\bicaption{基于STL分解的混合神经网络异常检测实验的整体漂移数据(左列)与斜率变化数据(右列)的原始遥测时序、降采样时序、分解后趋势项时序、周期项时序、残差项时序、异常分数时序与检测结果时序示意图}{Schematic illustration of the raw telemetry time series, downsampled time series, decomposed trend component, seasonal component, residual component, anomaly score time series, and detection result time series for overall drift data (left column) and slope change data (right column) in the STL decomposition–based hybrid neural network anomaly detection experiment.}
\label{fig:pic6}
\end{figure}

在包含早期离群点、中期漂移与晚期斜率变化的复杂场景中,本方法实现了对故障演化全周期的无缝跟踪。如图 \ref{fig:pic7} 所示,实验曲线准确描绘了故障发展的三个阶段:早期的尖峰报警、中期的平台期高位报警以及晚期的爬坡式报警。特别是得益于预设的权重分配策略,不同频段的异常特征在融合过程中未发生相互掩盖。实验结果证明,该多模型融合机制能够有效应对多故障并发的极端工况,实现了对复合故障模式的 100\% 覆盖。

\begin{figure}[htbp]
\centering
\includegraphics[width=0.6\linewidth]{figures/c4/pic7.png}
\bicaption{基于STL分解的混合神经网络异常检测实验的混合异常数据的原始遥测时序、降采样时序、分解后趋势项时序、周期项时序、残差项时序、异常分数时序与检测结果时序示意图}{Schematic illustration of the raw telemetry time series, downsampled time series, decomposed trend component, seasonal component, residual component, anomaly score time series, and detection result time series for mixed anomaly data in the STL decomposition–based hybrid neural network anomaly detection experiment.}
\label{fig:pic7}
\end{figure}

大规模测试集上的统计验证结果进一步确认了本方法在性能指标上的综合优势。实验数据显示,该方法对真实异常的召回率(Recall)超过 98.6\%,证明其能够极为全面地捕捉包括隐性趋势在内的绝大部分异常模式;同时,其检测结果的精确率(Precision)高达约 97.2\%。这两项核心指标的“双高”表现,证实了在引入STL分解与混合建模后,模型在大幅提升对隐性异常敏感度的同时,有效抑制了因噪声和周期干扰带来的误报风险。

\section{本章小结}

本章围绕星载设备高维遥测数据的异常检测与健康监测问题展开了系统研究。首先,阐述了星载设备在航天任务中的关键地位,将异常类型划分为设计缺陷、环境诱发、老化退化、随机故障及软件失误五大类,并重点聚焦于危害性大且隐蔽性强的“老化突发失效”与“未知硬件故障”。通过对现有基于统计学、机器学习及深度学习方法的梳理,指出了当前技术在应对高维、非线性、强噪声及多模式耦合数据时存在的局限性,特别是难以在故障早期实现有效预警的痛点。

针对上述挑战,本章提出了两类互补的无监督异常检测方法,并通过高保真仿真实验验证了其有效性。针对一般性高维遥测数据,提出了一种基于融合多窗口统计特征的LSTM自编码器检测方法。该方法通过引入多尺度滑动窗口提取统计特征,有效抑制了原始数据中的高频噪声;利用LSTM自编码器对特征序列进行时序建模,实现了对正常行为模式的无监督学习。实验结果表明,该方法在混合故障场景下表现出优异的抗噪性与鲁棒性,能够精准识别离群点、漂移及早期斜率变化,精确率高达99.81\%。

针对具有显著轨道周期特性的混合信号,提出了一种基于STL时序分解的LSTM-CNN混合自编码器检测架构。针对单一模型难以从强周期波动中剥离微弱趋势的难题,该方法利用STL算法将信号解耦为趋势、周期和残差分量,并分别采用LSTM、1D-CNN及统计模型进行“分而治之”的独立建模与加权融合。实验证实,该策略成功克服了周期信号的掩盖效应,显著提升了对隐性渐进式退化的检测灵敏度,在复杂复合故障检测中实现了超过98.6\%的查全率。

综上所述,本章所提出的方法体系有效解决了星载设备异常检测中面临的噪声干扰、时序依赖及特征耦合等核心问题,为实现卫星在轨的高可靠性自主健康管理提供了重要的理论依据与技术支撑。


























