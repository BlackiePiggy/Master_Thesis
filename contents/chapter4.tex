% !TEX root = ../main.tex

\chapter{星载产品异常机理及遥测多维信号特征分析}

\section{星载产品基本工作原理}

卫星作为在轨运行的高价值复杂系统,其健康状态的实时监测与管理对于保障任务成功、延长服务寿命至关重要。星载产品在轨期间会产生海量的多维遥测时序数据,这些数据是地面控制人员判断卫星健康状况的唯一依据。因此,从这些时序数据中准确、高效地检测出潜在或已发生的异常,是卫星健康管理领域的核心课题。

\subsection{传统异常检测方法及其在卫星领域的应用}

传统异常检测方法是该领域的早期技术基石,其核心思想通常是将异常定义为偏离已知正常模式的数据,并通过预设规则或统计分布来识别。

传统方法主要可分为基于统计学、基于距离/密度、基于模型等几类\cite{yu2022edge}。

基于统计学的方法假设正常数据遵循特定的统计分布。其中,阈值检测 (Threshold-based Techniques)是最简单直观的方法,通过设定固定阈值来判断遥测参数是否越界\cite{murphy2023overview}。例如,在北斗导航卫星系统(BDS)的星历数据异常检测中,就应用了总误差检测、一致性检查等基于阈值的技术\cite{cai2024evaluation}。统计模型如自回归积分滑动平均模型(ARIMA),通过预测时序数据的未来值,并将预测偏差作为异常判据\cite{murphy2023overview}。此外,Z-Score、中位数绝对偏差(MAD)和修正的汤普森陶氏检验(MTT)等也用于基于统计分布识别离群点\cite{bieber2023generic}。马氏距离 (Mahalanobis Distance)通过计算数据点到分布中心的距离,并考虑变量间的协方差,有效度量异常程度。由于其计算成本相对较低,被认为特别适用于资源受限的星载实时异常检测,有助于实现卫星的自我诊断能力\cite{katsube2025towards}。

基于距离和密度的方法包括k-近邻 (k-NN)、局部离群因子 (LOF)与核密度估计 (Kernel Density Estimates)等。k-近邻 (k-NN)通过计算数据点与其最近邻居的距离来识别异常\cite{bieber2023generic}。与其他点距离最大的点被认为是异常。局部离群因子 (LOF)通过比较一个数据点与其邻域的密度来识别局部异常,对于密度不均匀的数据集表现较好\cite{yu2022edge}。核密度估计 (Kernel Density Estimates)通过估计数据点的概率密度,将处于低密度区域的样本标记为异常\cite{guo2023contrastive}。

基于模型的方法中,有通过对卫星正常运行状态下的遥测数据建模,建立“健康基线”,任何偏离此基线的行为都被视为异常的方法。该方法已成功应用于卫星电源子系统的异常检测\cite{li2019research}。高斯混合模型 (GMM)作为一种聚类方法,GMM通过将数据拟合为多个高斯分布的混合,将低概率的样本识别为异常\cite{guo2023contrastive}。

传统方法因其原理直观、易于实现,在卫星健康管理中仍有应用,但存在高度依赖领域知识、处理高维复杂数据能力不足、难以识别未知异常、可伸缩性与在线更新困难等局限性。

为了克服传统方法的不足,研究界引入了更为强大的机器学习和深度学习技术。这些方法能够从数据中自动学习复杂的模式和特征,减少了对人工规则和领域知识的依赖。在深度学习全面普及之前,一些经典的机器学习算法被广泛应用于异常检测。

单类支持向量机 (OC-SVM)是一种半监督学习方法,OC-SVM仅在正常数据上进行训练,学习一个能够包围大部分正常数据的超球面或超平面边界,边界之外的数据点则被判定为异常\cite{bieber2023generic,yu2024amad}。

隔离森林 (Isolation Forest, iForest)是一种基于集成的无监督异常检测方法。它通过随机切分特征空间来构建多棵隔离树,异常点由于其“稀少且不同”的特性,通常在树的较浅层就能被孤立出来\cite{bieber2023generic}。

主成分分析 (PCA)是一种降维技术,其基本思想是将高维数据投影到低维空间,再通过重构误差的大小来判断数据点是否异常。重构误差大的点被认为是异常\cite{bieber2023generic}。

在处理极高维度、长时序依赖的复杂遥测数据时,传统机器学习的方法仍然可能面临性能瓶颈。深度学习方法,特别是深度神经网络,凭借其强大的特征学习和模式识别能力,已成为卫星遥测数据异常检测领域的研究热点。这些方法通常通过无监督或半监督的方式,学习正常数据的深层表示。

基于重构的方法以自编码器 (Autoencoder, AE)为核心展开。它由一个编码器和一个解码器组成,通过在正常数据上训练,学习将输入数据压缩成低维表示(编码)并从中重构出原始数据(解码)。对于异常数据,由于模型未曾学习过其模式,重构时会产生较大的误差,该重构误差可作为异常分数\cite{murphy2023overview}。变分自编码器 (Variational Autoencoder, VAE)作为AE的变种没,是一种生成模型,它学习数据的概率分布。例如,OmniAnomaly模型基于随机VAE,通过学习重构概率来检测异常,并结合POT(峰值超越阈值)方法自动选择阈值\cite{yu2024amad}。生成对抗网络 (Generative Adversarial Network, GAN)同样是一种生成模型,由一个生成器和一个判别器组成,通过相互博弈进行训练。在异常检测中,可以训练GAN来学习正常数据的分布。当一个新数据输入时,如果判别器能轻易地将其与生成器产生的正常数据区分开,则该数据可能为异常\cite{murphy2023overview}。

基于时序预测的模型通过各种模型输入。。。输出。。。,通过比较预测值与真实值得到误差值,如果误差值大于阈值则判别为异常。长短期记忆网络 (LSTM)作为一种循环神经网络(RNN),特别擅长处理时间序列数据,能够捕捉长期依赖关系。在异常检测中,LSTM可以被训练来预测下一个时间点的数据,如果实际观测值与预测值之间存在巨大差异,则判断为异常\cite{murphy2023overview}。例如,IF-TEA-LSTM模型就结合了iForest和LSTM,显著提升了北斗导航卫星系统星历数据的异常检测精度\cite{cai2024evaluation}。

近年来,Transformer架构因其在捕捉长程依赖方面的卓越能力,在时间序列分析领域取得了巨大成功。研究表明,基于Transformer的异常检测方法在处理卫星遥测数据时,性能优于传统的LSTM和AE模型,能够更好地捕捉数据中的复杂模式,并超越了在ESA OPS-SAT等公开数据集上的基准性能\cite{fejjari2025transformer}。

其他方面,对比学习是一种自监督学习方法,其目标是学习一个特征空间,在该空间中,相似的样本(如同一数据的不同增强视图)被拉近,不相似的样本被推远。CLPNM-AD (Contrastive Learning with Prototype-Based Negative Mixing for Anomaly Detection) 方法就是该思想在卫星遥测数据异常检测中的一个成功应用。该方法通过随机特征损坏生成增强样本,利用原型一致性策略捕捉样本的语义类别,并通过基于原型的负样本混合对比损失来学习正常模式的紧凑表示,从而更有效地分离正常与异常数据。实验表明,CLPNM-AD在多个卫星相关数据集上显著优于OC-SVM、DAGMM、Deep SVDD等基线方法,F1分数最高提升了11.5\%,并且对训练数据中的噪声污染具有良好的鲁棒性\cite{guo2023contrastive}。

\section{星载产品异常分类}

\section{遥测数据介绍}

\section{星载产品异常在遥测时序中的模式特征}

