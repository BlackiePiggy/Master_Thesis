% !TEX root = ../main.tex

\begin{abstract}[zh]
  全球导航卫星系统(Global Navigation Satellite System, GNSS)的现代化进程中伴随着多种影响系统完好性的GNSS事件。其中两类卫星端具有代表性且影响显著的事件分别是为增强抗干扰能力而主动实施的弹性功率操作,以及由老化或空间环境引发的星载设备被动隐性异常。当前针对此类事件的监测手段在检测性能与隐性异常识别能力上存在不足,且其对用户端精密定位的影响机制尚不明确,制约了高精度服务的可靠性。因此,本文利用多源时空信息,聚焦于这两类关键GNSS事件,从机理分析、异常检测到定位修正三个维度展开深入研究。本文的主要贡献和创新点包括:
  
  1)阐明了弹性功率的运行机理及其时空分布特征,揭示了其导致的载噪比变化与差分码偏差(Differential Code Bias, DCB)漂移规律,证实了该类事件是引起观测值异常和硬件偏差漂移的重要原因。 
  
  2)面向信号域,提出了基于动态时间规整的差分检测方法与基于深度学习的端到端检测方法。前者解决了“整体抬升”事件难以识别的难题,后者利用时空信息实现了无需历史基准的单历元高精度绝对状态判读,显著提升了检测的泛化性与实时性。 
  
  3)面向遥测域,针对高维遥测数据噪声干扰及强周期信号掩盖微弱趋势的问题,构建了融合多窗口统计特征与时序分解技术的混合神经网络模型。该方法有效解耦了周期项与趋势项,实现了对星载设备老化等隐性故障的早期精准预警。 
  
  4)建立了分段DCB修正策略并量化评估了弹性功率事件对精密单点定位(Precise Point Positioning, PPP)性能的影响。实验证明,该策略能显著缩短非差非组合及单频PPP的收敛时间并提升定位精度,有效解决了异常期间定位性能下降的问题。
\end{abstract}

\begin{abstract}[en]
  The modernization of the Global Navigation Satellite System (GNSS) is accompanied by various GNSS events that affect system integrity. Among these, two representative and far-reaching satellite-side events are the Flex Power operations actively implemented to enhance anti-jamming capabilities, and the passive implicit anomalies induced by onboard equipment aging or the space environment. Current monitoring methods for such events are insufficient in terms of real-time performance and the identification of implicit features. Furthermore, the mechanism of their impact on user-side precise positioning remains unclear, limiting the reliability of high-precision services. Therefore, utilizing multi-source spatiotemporal information, this thesis focuses on these two critical GNSS events and conducts in-depth research across three dimensions: mechanism analysis, anomaly detection, and positioning correction. The main contributions and innovations of this thesis include:
  \begin{enumerate}
    \item The operating mechanism and spatiotemporal distribution characteristics of Flex Power are elucidated. The research reveals the laws of Carrier-to-Noise density ratio ($C/N_0$) steps and intra-day drifts of Differential Code Bias (DCB) caused by Flex Power, confirming that such events are significant physical inducements for observation anomalies and hardware bias instability.
    
    \item For signal domain detection, a differential detection method based on Dynamic Time Warping (DTW) and an end-to-end detection method based on deep learning are proposed. The former solves the difficulty in identifying non-step "overall lift" events, while the latter achieves high-precision absolute state interpretation at a single epoch without historical baselines, significantly improving detection generalization and real-time performance.
    
    \item For telemetry domain monitoring, addressing the issues of noise interference in high-dimensional telemetry data and strong periodic signals masking weak trends, a hybrid neural network model fusing multi-window statistical features and time-series decomposition techniques is constructed. This method effectively decouples periodic and trend components, achieving early and precise warning for implicit faults such as satellite equipment aging.
    
    \item The impact of Flex Power events on the performance of Precise Point Positioning (PPP) is quantitatively assessed, and a segmented DCB correction strategy is established. Experiments demonstrate that this strategy significantly shortens the convergence time of dual-frequency and single-frequency PPP and improves positioning accuracy, effectively resolving the issue of positioning divergence during anomaly periods.
  \end{enumerate}
  
\end{abstract}
