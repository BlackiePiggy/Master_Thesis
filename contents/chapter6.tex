% !TEX root = ../main.tex

\chapter{总结与展望}

\section{全文总结}
GNSS 系统的现代化发展在带来更高服务精度的同时,也引入了如弹性功率增强及复杂的星载设备隐性故障等 GNSS 事件。这些事件导致的信号特征异常变化和硬件偏差漂移,对高精度定位服务的完好性与可靠性构成了挑战。本文旨在利用多源时空信息,建立一套从机理分析、异常检测到用户端影响评估与修正的完整体系。围绕这一核心目标,本文深入剖析了弹性功率的运行机理及其导致的 C/N0 与 DCB 时空变化特征,分别在信号域和遥测域提出了基于传统统计与深度学习的异常检测方法,并量化评估了此类事件对 PPP 的影响,提出了相应的修正策略。本文的主要研究成果总结如下:

1) 揭示了 GNSS 弹性功率的运行机理及其对观测数据的时空影响特征。

针对弹性功率事件缺乏官方公开信息且影响机制不明的问题,本文系统梳理了 GPS 与 BDS 弹性功率的定义、历史演变及运行模式。同时,工作可视化分析了弹性功率开启时的 C/N0 变化情况,归纳出“阶跃提升”与“整体提升”两类变化特征,量化研究了不同接收机与天线组合对信号响应的系统性差异,指出不同天线组间的平均 C/N0 提升值差异可达 1.53 dB-Hz,最大差异达 2.56 dB-Hz。深入剖析了弹性功率对卫星硬件通道特性的影响,阐明了信号物理层变化导致卫星 DCB 发生日内跳变的机理。通过构建短滑动窗口 DCB 估计算法,量化了 GPS 卫星在弹性功率器期间的DCB变化情况。频内 DCB 系统性偏移量约为 0.40 ns,估计精度 RMS 小于 0.0075 ns,证实其是受弹性功率直接影响的主要对象;频间 DCB 估计精度 RMS 在 0.050 ns水平,未检测到明显的阶跃偏差,主要受电离层周日变化主导。结果为后续针对性修正提供了理论依据为后续检测与修正提供了高精度的理论依据与数据支撑。

2) 提出了信号域基于时序匹配差分与深度学习的弹性功率事件检测方法。

针对传统 FPD 算法难以检测非阶跃型事件且依赖大规模测站网的局限,本文提出了两种创新检测方法。首先,提出了基于动态时间规整的差分检测方法 AFPD-DTW,利用序列匹配思想解决了观测序列与基准模型间的时间错位问题,在仅需 8-10 个测站的情况下,实现了 99.86\% 的平均召回率与 99.98\% 的精确率,且检测速度较 FPD 方法提升约 20 倍。进一步地,为实现无需历史基准的绝对检测,提出了基于深度学习的端到端检测方法 DLFPD。该方法构建了融合 CNN 与 Transformer 的混合神经网络架构,利用归一化信号强度及空间几何特征,实现了单测站、单历元的高精度状态判断。实验表明,DLFPD 方法在测试集上的 F1 分数高达 99.96\%,在最严苛的时空双重泛化测试下准确率仍保持在 99.73\%,且对 6 颗受测卫星实现了 100\% 的召回率。

3) 构建了遥测域面向星载设备隐性故障的多维时序检测模型。

针对星载设备老化突发失效及未知硬件故障隐蔽性强、早期特征微弱的问题,本文从遥测数据挖掘角度提出了解决方案。针对一般性高维遥测数据,提出了融合多窗口统计特征的 LSTM 自编码器方法,通过多尺度特征提取有效抑制高频噪声并捕捉不同尺度时序关联性,在约十万样本混合故障场景下的精确率高达 99.81\%,F1 分数达到 98.47\%,误报数仅为 6 例。针对具有强周期性的遥测参数,提出了基于 STL 时序分解的 LSTM-CNN 模型。该模型通过解耦趋势项、周期项与残差项并分而治之,克服了强周期信号对微弱趋势异常的掩盖效应。基于仿真数据的实验验证表明,该方法在复杂混合故障场景下实现了超过 98.6\% 的召回率与 97.2\% 的精确率,能够有效实现对星载设备全生命周期隐性故障的早期精准预警。

4) 建立了DCB分段估计策略并量化评估了弹性功率对 PPP 的影响。

针对卫星端硬件偏差漂移如何影响用户端定位性能的问题,本文推导了从双频IF组合到非差非组合再到SF- PPP 的观测方程,明确了 DCB 参数在不同函数模型中的误差传递路径。通过设计单站与全球多站的对比实验,利用实测数据全方位量化了影响。在MAL2单测站情况下,使用错误未分段和正确分段的 DCB 修正产品,UC-PPP 的收敛时间分别为 4170 s 和 1440 s(提速近 3 倍),东向定位精度分别为 0.1201 m 和 0.0283 m,精度相差约 76\%;SF- PPP 的收敛时间则分别为 15450 s 和 2370 s,收敛速度相差约 6.5 倍,且错误修正下的东向精度仅为 0.2876 m。在全球多站统计层面,使用错误未分段和正确分段的 DCB 修正产品,UC-PPP 的全球平均收敛时间分别为 80.5 分钟和 55.4 分钟,三维定位精度分别为 0.223 m 和 0.153 m,精度提升约 31\%;对于对偏差更为敏感的SF- PPP,两者的全球平均收敛时间分别为 128.3 分钟和 41.5 分钟(提速超 200\%),三维定位精度分别为 0.462 m 和 0.362 m,精度提升约 21\%。这些结果证实了该策略能有效保障弹性功率期间的高精度定位服务性能。

\section{研究展望}
尽管本文在 GNSS 事件机理分析、异常检测及定位影响修正方面取得了一定成果,但受限于数据获取条件和研究时间的限制,针对部分研究点仍有进一步深入挖掘的空间。具体来说,可针对以下几点展开下一步工作:

1) 扩展 GNSS 事件对精密定位及其他参数的综合影响评估

本文目前的研究主要聚焦于弹性功率对 UC-PPP 以及 SF-PPP 定位收敛及精度的影响,且分析对象主要集中在 DCB 参数。未来的工作应将评估范围从 PPP 扩展至PPP-AR,量化分析弹性功率引发的偏差跳变对模糊度固定成功率及固定后残差的具体影响。同时,除了 DCB,还应深入探究事件对相位小数偏差、电离层建模参数等更多关键观测量的影响机制。此外,应涵盖更多频段的定位性能评估,深入剖析不同频段受影响程度差异的内在机理,探究其是否与信号调制方式、相关器间距或特定频段的抗干扰设计有关,从而建立更精细化的频段差异化修正模型。

2) 提升频间 DCB 估计精度以精细化体现事件特征

本文研究发现,受限于当前的估计精度,难以从频间 DCB 中显著提取出弹性功率引起的微小变化。未来需致力于突破高精度频间 DCB 估计的瓶颈,以探明弹性功率是否对频间偏差存在影响。具体的方向应该包括:引入多层电离层模型或外部高精度 GIM 约束以削弱电离层残差干扰;优化接收机端偏差与卫星端偏差的分离策略,例如采用零基线或短基线差分消除接收机端影响;以及利用更高采样率的观测数据进行噪声平滑,从而捕捉可能被掩盖的频间偏差微小漂移特征。

3) 探究大规模测站网络下的定位性能影响因素

本文对 PPP 定位影响的评估目前仅基于 10 个全球测站的平均统计结果,尚未充分揭示影响的空间分布规律。下一步工作应将评估规模扩展至百余个全球分布的 MGEX 测站,从平均影响转向空间相关性的研究。重点应该分析导致不同测站定位性能衰减差异的深层因素,例如测站所处的经纬度、接收机硬件差异、以及事件发生期间卫星相对于测站的空间几何构型。通过建立测站特征与定位误差的关联模型,为不同区域、不同硬件配置的用户提供更具针对性的完好性预警方法。

4) 验证真实在轨遥测数据并增强深度学习模型的可解释性

本文在星载设备异常检测中使用的是基于真实遥测特征的仿真数据,尽管模拟了典型故障,但真实在轨环境更为复杂,存在多种未知的耦合故障模式与非典型环境噪声。未来的研究应致力于获取真实的卫星在轨遥测数据,从仿真到现实以验证模型在应对多种未知异常类型时的泛化能力。同时,针对深度学习模型存在的黑盒问题,未来应引入可解释性分析技术,如注意力机制可视化或梯度加权类激活映射,以明确模型判定异常的具体依据,使地面运控人员能够理解故障背后的物理逻辑,提升系统判断的可信度。

5) 构建跨星座、更高泛化性的 GNSS 事件通用检测模型

目前本文提出的基于深度学习的弹性功率检测模型主要针对 GPS 系统且集中在单一星座的泛化。考虑到 BDS 等其他星座同样面临潜在的信号异常风险,未来的研究可探索构建跨星座的通用基础模型。尝试将 GPS 与 BDS 的信号特征与数据进行融合,利用多任务学习或迁移学习技术,训练一个能够同时适应不同星座信号体制的通用检测网络。此外,也应继续提升模型的可解释性,分析模型在跨星座迁移中提取的共性特征与异性特征,最终实现对多系统 GNSS 弹性事件的单模型自动化监测。